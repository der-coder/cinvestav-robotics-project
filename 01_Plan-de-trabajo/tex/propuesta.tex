\section{Introducción}

El documento presente contiene el plan de trabajo elaborado por
los miembros del equipo 4 (cuatro), conformado por 
Enrique Benavides Tellez, Isaac Ayala Lozano y 
Neftalí Jonatán González Yances para la realización del proyecto 
descrito en el documento \emph{Manual de Proyecto del 
curso de Robótica I} para
el cuatrimestre Septiembre - Diciembre 2019. 
El proyecto consiste en la comparación del desempeño y gasto energético de
diferentes leyes de control clásicas en el simulador del robot
paralelo tipo Gough-Stewart. 

\section{Plan de trabajo}

A continuación se presentan las metodologías y horarios para el desarrollo del 
proyecto. Se propone lo siguiente:

\begin{itemize}
 \item Las actividades semanales para el avance del proyecto se dividirán entre 
los miembros del equipo de acuerdo a las habilidades de cada integrante. A 
manera de facilitar su desarrollo e integración con el proyecto en general.
\item Los documentos electrónicos se albergarán en un repositorio compartido el 
cual mantendrá la versión más actualizada de todo el trabajo del proyecto.
\item El repositorio contendrá 4 (cuatro) ramas de desarrollo principales, 
teniendo una rama \textbf{master} que contendrá los avances aprobados del 
proyecto por el equipo.
\subitem Cada integrante contará con una rama de desarrollo personal y será 
responsable de la misma, así como de mantener su rama personal actualizada con 
la rama \textbf{master}.
\subitem Cualquier aportación a la rama \textbf{master} se efectuará mediante 
\textbf{Pull Requests} al repositorio para preservar la historia de los avances 
y como Control de Calidad para el proyecto.

\end{itemize}

\subsubsection{Horario y actividades planeadas}

La figura
% \ref{fig: horario} 
presenta los horarios semanales agendados 
para las actividades del curso y proyecto. En caso de necesitar más tiempo de 
trabajo para alguna actividad cualquiera, el equipo asignará tiempo adicional 
según se requiera.


