\section{Introducción}

El documento presente contiene el plan de trabajo elaborado por
los miembros del equipo 4 (cuatro), conformado por 
Enrique Benavides Tellez, Isaac Ayala Lozano y 
Neftalí Jonatán González Yances para la realización del proyecto 
descrito en el documento \emph{Manual de Proyecto del 
curso de Robótica I} para
el cuatrimestre septiembre - diciembre 2019. 
El proyecto consiste en la comparación del desempeño y gasto energético de
diferentes leyes de control clásicas en el simulador del robot
paralelo tipo Gough-Stewart. 

\section{Plan de trabajo}

A continuación se presentan las metodologías y horarios para el desarrollo del 
proyecto. Se propone lo siguiente:

\begin{itemize}
 \item Las actividades semanales para el avance del proyecto se dividirán entre 
los miembros del equipo de acuerdo a las habilidades de cada integrante. La 
tabla \ref{table: abilities} presenta las habilidades de cada integrante 
relevantes para el proyecto.
\item Cada integrante tendrá un espacio de desarrollo independiente en línea y 
se mantendrá una versión de desarrollo principal a la que cada uno aportará sus 
avances de manera controlada.


\end{itemize}
 

\begin{table}[h!]
\centering
\begin{tabular}{l|c|c|c}
Software & Enrique & Jonatán & Isaac\\
\hline
CAS  & Scilab & Matlab & Mathematica\\
CAD & Inventor & Solidworks  & NX\\
Lenguajes de programación & Python & M & JavaScript

\label{table: abilities}

\end{tabular}
\caption{Habilidades de los integrantes.} 
\end{table}


\subsubsection{Horario y actividades planeadas}

La figura
% \ref{fig: horario} 
presenta los horarios semanales agendados 
para las actividades del curso y proyecto. En caso de necesitar más tiempo de 
trabajo para alguna actividad cualquiera, el equipo asignará tiempo adicional 
según se requiera.


\begin{table}[h!]
\centering
\begin{tabular}{|c|c|}
\label{table: calendar}
Fecha & Actividad\\
\hline
25-oct & Simulador en 
01-nov & Simulador terminado: animación y resultados.\\
06-nov & Evaluación de resultados del simulador.\\
11-nov & Manual en desarrollo, reporte final en desarrollo.\\
18-nov & Manual terminado, reporte final en revisión, simulador terminado.\\
25-nov & Reporte terminado, presentación en desarrollo.\\
29-nov & Presentación terminada.\\
\hline
\end{tabular}
\caption{Calendario.} 
\end{table}
