\documentclass{beamer}

\usepackage{import}
\usepackage{xifthen}
\usepackage{pdfpages}
\usepackage{transparent}


% This file is a solution template for:

% - Talk at a conference/colloquium.
% - Talk length is about 20min.
% - Style is ornate.



% Copyright 2004 by Till Tantau <tantau@users.sourceforge.net>.
%
% In principle, this file can be redistributed and/or modified under
% the terms of the GNU Public License, version 2.
%
% However, this file is supposed to be a template to be modified
% for your own needs. For this reason, if you use this file as a
% template and not specifically distribute it as part of a another
% package/program, I grant the extra permission to freely copy and
% modify this file as you see fit and even to delete this copyright
% notice. 


\mode<presentation>
{
  \usetheme{Berlin}
  \setbeamercovered{transparent}
}


\usepackage[spanish,mexico]{babel}
\usepackage[utf8]{inputenc}
\usepackage{times}
\usepackage[T1]{fontenc}

\title{Plataforma Gough-Stewart}

\subtitle
{Reporte de medio término}

\author[]{E. Benavides \and I. Ayala \and N. González}


\institute[]
{
  Centro de Investigación y de Estudios Acanzados del IPN\\
  Robótica y Manufactura Avanzada
  }

\date[]{RYMA 2019}

\subject{Robotics}



% Delete this, if you do not want the table of contents to pop up at
% the beginning of each subsection:
% \AtBeginSubsection[]
% {
%   \begin{frame}<beamer>{Outline}
%     \tableofcontents[currentsection,currentsubsection]
%   \end{frame}
% }


% If you wish to uncover everything in a step-wise fashion, uncomment
% the following command: 

%\beamerdefaultoverlayspecification{<+->}


\begin{document}

\begin{frame}
  \titlepage
\end{frame}

\begin{frame}{Contenido}
  \tableofcontents
\end{frame}


\section{Introducción}

\subsection{Motivación}

\begin{frame}{Uso de la plataforma Gough-Stewart}

  \begin{itemize}
    \item Movimiento preciso de objetos 
    \item Posicionamiento de antenas parabólicas
  \end{itemize}
  
\end{frame}

\section{Desarrollo}

\subsection{Cinemática}

\begin{frame}{Abstracción del modelo}{}
 \begin{center}
 \begin{figure}
 \import{../img/}{goughStewart.pdf_tex} 
 \caption{Abstracción de geometría.}
 \label{key: diagram}
 \end{figure}

 
 \end{center}

\end{frame}

\begin{frame}{Relación de posiciones}

\begin{itemize}
 \item Se establece la relación de posición entre la base y la plataforma\\
 \begin{equation}
  p_i = d + Ra_i = b_i + l_i
  \label{eq: position}
 \end{equation}

 
\end{itemize}
\end{frame}

\begin{frame}{Transformaciones homogéneas}
\begin{itemize}
 \item Definimos $R$ como la matriz de rotación extrínseca de la plataforma respecto a la base. \\
 \begin{equation}
R = R_zR_yR_x = R_{xyz}
\end{equation}
\end{itemize}
\end{frame}

\begin{frame}{Pistones}
\begin{itemize}
 \item La ecuacion \ref{eq: position} se expresa en función de la longitud $l_i$ de cada pistón
 \begin{equation}
  l_i = d + Ra_i - b_i
 \end{equation}
\end{itemize}

\end{frame}

\begin{frame}{Coordenadas generalizadas}
 \begin{itemize}
  \item Designamos $||l_i||$ como las coordenadas generalizadas $q_i$
  
  \begin{equation}\label{eq_coordgral}
q_i = ||l_i|| = \sqrt{l_i^Tl_i}
\end{equation}
  
 \end{itemize}

\end{frame}


\begin{frame}{Jacobiano inverso}
\begin{itemize}
 \item Dada la siguiente ecuacion
 
 \begin{equation}
\dot{q}=J^{-1} \nu = A \begin{bmatrix}
v_p\\
\omega
\end{bmatrix} \Rightarrow A = J^{-1}
\end{equation}

\item Se plantea

\begin{equation}
\dot{q}=\frac{1}{||l_i||}(\dot{d} + [\omega \times] Ra_i)\cdot(d + Ra_i -b_i) 
\end{equation}

\end{itemize}

 
\end{frame}

\begin{frame}{Jacobiano inverso}

\begin{itemize}
 \item Se obtiene
\begin{equation} 
\label{eq: inverse jacobian}
\dot{q} = \frac{1}{||l_i||} [l_i^T , [(Ra_i)\times]l_i^T] \begin{bmatrix}
v_p\\
\omega
\end{bmatrix}
\end{equation}
 
\end{itemize}



\end{frame}

\begin{frame}{Jacobiano}
 \begin{itemize}
  \item De la ecuación \ref{eq: inverse jacobian} se llega a 
  
  \begin{equation}\label{jac_A}
        A = J^{-1} = 
        \begin{bmatrix}
            \vec{u_i}^T & [(Ra_i)\times]\vec{u_i}^T
        \end{bmatrix}
\end{equation}

\item Dado que $J = A ^{-1}$

\begin{equation*}
J = \begin{bmatrix}
\vec{u_i}^T & [(Ra_i)\times]\vec{u_i}^T
\end{bmatrix}^{-1}
\end{equation*}
 \end{itemize}

\end{frame}

\subsection{Energía}
\subsection{Energía cinética}

\begin{frame}{Energía cinemática del sistema}
 \begin{equation*}
\begin{bmatrix}
\vec{u_i}^T & [(Ra_i)\times]\vec{u_i}^T
\end{bmatrix}^{-1} \dot{q} = \begin{bmatrix}
v_p\\
\omega
\end{bmatrix}
\end{equation*}
\end{frame}

\section{Simulador}
\subsection{MATLAB}
\begin{frame}{GSP Toolbox}
 
\end{frame}

\begin{frame}{Interfaz gráfica}
 
\end{frame}




\end{document}


