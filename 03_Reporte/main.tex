\documentclass[techreport]{IEEEtran}
\usepackage{cite}
\usepackage{amsmath,amssymb,amsfonts}
\usepackage{algorithmic}
\usepackage{graphicx}
\usepackage{textcomp}
\usepackage{xcolor}
\usepackage{listings}
\usepackage[spanish,mexico]{babel}

\usepackage{caption}
\usepackage{subcaption}

\usepackage{import}
\usepackage{xifthen}
\usepackage{pdfpages}
% \usepackage{multicol}
% \usepackage{transparent}


\def\BibTeX{{\rm B\kern-.05em{\sc i\kern-.025em b}\kern-.08em
    T\kern-.1667em\lower.7ex\hbox{E}\kern-.125emX}}
    
\begin{document}

\title{
Análisis, simulación y comparación de gastos energéticos de la plataforma Gough-Cappel con distintos modelos de control\\
}


\author{
    \IEEEauthorblockN{
    Enrique Benavides Téllez, 
    Isaac Ayala Lozano y 
    Neftali Jonatán González Yances\\
    }
    \IEEEauthorblockA{
    \textit{Robótica y Manufactura Avanzada} \\
    \textit{CINVESTAV}\\
        Ramos Arizpe, México}
}

\maketitle

\begin{abstract}
Se presenta la comparación energética de tres estrategias de control en lazo cerrado para la plataforma Gough-Cappel.
Se describe el desarrollo matemático del sistema y la implementación del mismo en MATLAB.
\end{abstract}

% % \begin{IEEEkeywords}
% \end{IEEEkeywords}


\section{Introducción}

La plataforma Gough-Cappel es un robot paralelo de seis grados de libertad.
Está compuesto por seis actuadores prismáticos que conectan una plataforma
móvil con una base mediante cardanes en un extremo de los actuadores y por
juntas esféricas o cardanes en el extremo contrario. 

% TODO image here

Conocido en la literatura como la 
plataforma Gough-Stewart, su origen se remonta a mediados del sigo XIX.
Su invención está documentada en 1954 por Eric Gough 
\cite{gough} en Inglaterra como una máquina para 
evaluar la respuesta de llantas de aviones a diferentes 
condiciones de aterrizaje (figura). 
De manera independiente, Klaus Cappel inventó un simulador de vuelo (figura )
con la misma arquitectura y una solicitud de patente 
fue sometida en 1964.
La popularización de este diseño de robot es debido a D. Stewart 
por su publicación en 1965, que sorprendentemente presenta el
diseño de un simulador de vuelo que no asemeja a la 
plataforma Gough Stewart (figura ).

% http://www.parallemic.org/Reviews/Review007.html

% TODO Insert 3 images for the designs by Gough, Cappel, and Stewart

\begin{figure}[hb!]
    \centering
    \begin{subfigure}[b]{0.3\textwidth}
%         \includegraphics[width=\textwidth]{case03_phase_r_d_r}
        \caption{Máquina universal diseñada por Gough.}
        \label{fig: gough robot}
    \end{subfigure}
    ~ %add desired spacing between images, e. g. ~, \quad, \qquad, \hfill etc. 
      %(or a blank line to force the subfigure onto a new line)
    \begin{subfigure}[b]{0.3\textwidth}
%         \includegraphics[width=\textwidth]{case03_phase_theta_d_theta}
        \caption{Simulador de vuelo patentado por Cappel.}
        \label{fig: cappel robot}
    \end{subfigure}
    ~ %add desired spacing between images, e. g. ~, \quad, \qquad, \hfill etc. 
    %(or a blank line to force the subfigure onto a new line)
    \begin{subfigure}[b]{0.3\textwidth}
%         \includegraphics[width=\textwidth]{case03_phase_theta_d_theta}
        \caption{Simulador de vuelo propuesto por Stewart.}
        \label{fig: stewart robot}
    \end{subfigure}
    \caption{Iteraciones del robot paralelo.}\label{fig: parallel robots}
\end{figure}


\section{Marco teórico}

\subsection{Mecánica Lagrangiana}

asd

\subsection{Formulación de Kirchhoff para la dinámica de un cuerpo rígido}

De

La PGS puede ser modelada en un inicio por medio de su cinemática inversa. Observando el modelo en la figura (insertar figura de la plataforma) se puede formular lo siguiente:
\begin{equation} \label{plat_grl}
p_i = d + Ra_i = b_i + l_i
\end{equation}
<<<<<<< HEAD
Siendo $p_i$ el lugar donde el actuador de la plataforma es colocado, el actuador es un piston el cual es controlado por el largo. Tomando en cuenta que el valor del actuador es $l_i$, podemos reescribir la ecuación \ref{plat_grl} de la siguiente manera.
\begin{equation}
l_i = d + Ra_i - b_i
\end{equation}
Sin embargo el valor $l_i$ es un vector en coordenadas $[x y z]^T$ al cual le debemos aplicar la norma para obtener el valor de dimensión del actuador y ese valor será la coordenada generalizada de la plataforma.
\begin{equation}\label{eq_coordgral}
q_i = ||l_i|| = \sqrt{l_i^Tl_i}
\end{equation}
Con esto se tiene el valor de las coordenadas generalizadas de un punto y orientación deseada. De esta ecuación se obtendrá el jacobiano para despejar el twist y poder utilizar los valores de velocidad lineal y angular de la plataforma para el desarroollo del control por fuerzas. Para obtener el jacobiano se plantea la siguiente ecuación
\begin{equation} \label{equgral_q}
J\dot{q}=\nu
\end{equation}
La ecuación \ref{eq_coordgral} al ser derivada respecto al tiempo se parece a la ecuacion del jacobiano en la solución de las velocidades de las coordenadas generalizadas.
=======
Siendo $Ra_i$ el lugar donde un extremo del actuador en la plataforma es colocado, el valor $b_i$ es el lugar donde el otro extremo del actuador es colocado en la base y $d$ es la distancia que debe de moverse la plataforma respecto de la base. Definimos $R$ como la matriz de rotación extrínseca de la plataforma respecto a la base. 
\begin{equation}
R = R_zR_yR_x = R_{xyz}
\end{equation}
El actuador es un pistón controlado por el largo y tomando en cuenta que el valor del actuador es $l_i$, podemos reescribir la ecuación \ref{plat_grl} de la siguiente manera.
\begin{equation}
l_i = d + Ra_i - b_i
\end{equation}
El vector $l_i$ se obtiene como coordenadas $[x\ y\ z]^T$ al cual se debe aplicar la norma para obtener la dimensión del actuador y será la coordenada generalizada del pistón i-ésimo de la plataforma.
\begin{equation}\label{eq_coordgral}
q_i = ||l_i|| = \sqrt{l_i^Tl_i}
\end{equation}
De la ecuación (\ref{eq_coordgral}) se obtendrá el jacobiano para después despejar el twist y utilizar los valores de velocidad lineal y angular de la plataforma para el desarrollo del control por fuerzas. Para obtener el jacobiano se plantea la siguiente ecuación
\begin{equation} \label{equgral_q}
J\dot{q}=\nu
\end{equation}
La ecuación \ref{eq_coordgral} al ser derivada respecto al tiempo es parecido a la ecuación del jacobiano en la solución de las velocidades de las coordenadas generalizadas.
>>>>>>> isaac
\begin{equation}
\frac{d}{dt}q = \frac{d}{dt}||l_i|| = \frac{d}{dt}\sqrt{l_i^Tl_i} 
\end{equation}
La ecuación anterior se desarrolla para tener la forma del jacobiano inverso:
\begin{equation}
\dot{q}=J^{-1} \nu = A \begin{bmatrix}
v_p\\
\omega
\end{bmatrix} \Rightarrow A = J^{-1}
\end{equation}

Desarrollamos la derivada de $||l_||i$:
\begin{equation}
\dot{q} = \frac{1}{2||l_i||} \dot{l_i} \cdot l_i + l_i \cdot \dot{l_i} = \frac{1}{||l_i||} \dot{l_i} \cdot l_i \\ \bigskip
\end{equation}

\begin{equation}
\dot{q}=\frac{1}{||l_i||}(\dot{d} + [\omega \times] Ra_i)\cdot(d + Ra_i -b_i) 
\end{equation}
\begin{equation*}
= \frac{1}{||l_i||}(v_p - [(Ra_i)\times]\omega)(l_i)
\end{equation*}

\begin{equation}
\dot{q} = v_p \cdot l_i - [(Ra_i)\times]\omega \cdot l_i 
\end{equation}
\begin{equation*}
= v_p \cdot l_i + [(Ra_i)\times]l_i \cdot \omega
\end{equation*}

\begin{equation}
\dot{q} = \frac{1}{||l_i||} v_p \cdot l_i + [(Ra_i)\times]l_i \cdot \omega
\end{equation}
\begin{equation} \label{jac_inv}
\dot{q} = \frac{1}{||l_i||} [l_i^T , [(Ra_i)\times]l_i^T] \begin{bmatrix}
v_p\\
\omega
\end{bmatrix}
\end{equation}
Con el desarrollo se encuentra que la jacobiana inversa parte de la ecuación \ref{jac_inv} y se define como:
\begin{equation}\label{jac_A}
A = J^{-1} = \begin{bmatrix}
\vec{u_i}^T & [(Ra_i)\times]\vec{u_i}^T
\end{bmatrix}
\end{equation}
Al invertir la matriz $A$ de la ecuación \ref{jac_A} se obtiene la jacobiana de la PGS, con la cual se obtendrán las velocidades lineales y angulares de la PGS.
\begin{equation*}
J = \begin{bmatrix}
\vec{u_i}^T & [(Ra_i)\times]\vec{u_i}^T
\end{bmatrix}^{-1}
\end{equation*}
Utilizando la Jacobiana y desarrollando la ecuación \ref{equgral_q} se obtienen las velocidades lineales y angulares de la PGS.
\begin{equation*}
\begin{bmatrix}
\vec{u_i}^T & [(Ra_i)\times]\vec{u_i}^T
\end{bmatrix}^{-1} \dot{q} = \begin{bmatrix}
v_p\\
\omega
\end{bmatrix}
\end{equation*}
Para el desarrollo de la dinámica del robot, se conoce que soportara en la plataforma un disco parabólico con masa de 250 lb. Para obtener el modelo dinámico de la PGS con el disco hay se necesitan los valores de velocidad lineal y angular con respecto de la plataforma, se plantea la siguiente transformación.
\begin{equation} \label{equ_tr0-1}
\begin{split}
M_{6x6}\
\begin{bmatrix}
v_p^0\\
\omega^0
\end{bmatrix}  = \begin{bmatrix}
v_p^1\\
\omega^1
\end{bmatrix}
\end{split}
\end{equation}
En la ecuación \ref{equ_tr0-1} se identifica que al multiplicar las velocidades respecto a la base de la PGS se obtienen los valores de velocidades respecto a la plataforma. La matriz $M$ debe de tener la siguiente forma:
\begin{equation}
M = \begin{bmatrix}
I & 0_{3x3} \\
0_{3x3} & R^T
\end{bmatrix}
\end{equation}
Donde el primer bloque es la matriz identidad debido a que las velocidades lineales tanto de la base como de la plataforma son iguales, la matriz $E$ es la transformación de las velocidades angulares y tiene la siguiente forma:
\begin{equation*}
R^T = \begin{bmatrix}
R_{11} & R_{21} & R_{31}\\
R_{12} & R_{22} & R_{32}\\
R_{13} & R_{23} & R_{33}
\end{bmatrix}
\end{equation*}
Con la obtención del twist respecto a la plataforma del robot se pueden desarrollar las ecuaciones de energía. Se obtiene la energía potencial por medio de la altura del disco parabólico $dp$ sobre la plataforma del robot.
\begin{equation}
\begin{split}
P_{dp} = m_{dp}gh = [0\ 0\ m_{dp}g] \begin{bmatrix}
P_x\\
P_y\\
P_z\\
\end{bmatrix}\\
\\
P_z = d + R\begin{bmatrix}
0\\
0\\
cm_{dp}\\
\end{bmatrix}\\
\end{split}
\end{equation}
El punto $P_z$ es la altura del disco parabólico respecto a la base de la PGS. La energía cinética depende de las velocidades sobre el disco parabólico y se define
\begin{equation}
K = \frac{1}{2} \nu m_{dp} \nu = \frac{1}{2} \dot{q}^T J^T\ m\ J \dot{q}
\end{equation}

El modelo cinemático desarrollado en la sección
 fue implementado en MATLAB
para evaluar el comportamiento del sistema.
Este simulador es capaz de calcular los cambios en 
la dimensión de cada elemento de movimiento del sistema
para lograr que la plataforma móvil 
llegue a la posición deseada.

El simulador cuenta con una interfaz gráfica que permite la 
evaluación de coordenadas y orientación de la plataforma.
Es posible especificar posiciones deseadas y 
los parámetros de movimiento del robot como lo son la
longitud de cada pistón y la dimensión de la base y 
plataforma del sistema.

\begin{figure}
 \centering
 \includegraphics[scale=0.2]{img/principal.png}
 % principal.png: 1812x922 px, 120dpi, 38.36x19.52 cm, bb=0 0 1087 553
 \caption{Interfaz gráfica del simulador.}
 \label{fig: GUI}
\end{figure}


El simulador determina si la posición y 
orientación introducidas en la interfaz gráfica
es realizable con los parámetros especificados para
el robot. Para llegar a este resultado, el simulador realiza
las siguientes instrucciones:

\begin{itemize}
 \item Obtener la posición de cada junta de la base,
 con respecto al centro de la base.
 \item Obtener la posición de cada junta de la
 plataforma móvil con respecto a la base y tomando
 en cuenta la posición deseada.
 \item Procesar las coordenadas de cada elemento para 
 obtener los vectores que representan cada pistón.
 \item Evaluar si la posición deseada excede las 
 limitaciones físicas del sistema como exceder la 
 extensión o compresión máxima de algún pistón.
 \item Enviar mensajes de error si es necesario.
 \item Graficar los datos obtenidos, en caso de ser una
 posición factible.
 \item Ejcutar instrucciones adicionales 
 de la interfaz gráfica.
\end{itemize}

El simulador también es capaz de producir animaciones del 
movimiento de la plataforma y exportarlas como 
archivos MP4.


El proyecto ha logrado parametrizar 
en una medida considerable la simulación de la 
plataforma Gough Stewart, con lo que se vuelve fácil 
evaluar una infinidad de variaciones del sistema.
Esta parametrización ya ha demostrado sus ventajas 
en la etapa de pruebas de la interfaz gráfica.
La posibilidad de estudiar el comportamiento del
sistema a diferentes escalas ha permitido 
identificar errores en el comportamiento del 
simulador y corregir dichos errores de manera adecuada.


\section{Discusión}

\subsection{Entendimiento del problema}

\subsection{Simulador en lazo abierto}

\subsection{Implementación de estrategias de control}



\subsection{Implementación de resortes de fin de carrera}

\section{Conclusiones}

\subsection{Neftali Jonatán González}

\subsection{Enrique Benavidez Téllez}

\subsection{Isaac Ayala Lozano}


\bibliographystyle{IEEEtran}
\bibliography{bibliografia.bib}


\appendix

\section{Momentos de inercia}

Los momentos de inercia del sistema fueron obtenidos
empleando la herramienta de Solidworks. 
Las propiedades de densidad de los componentes fueron asignados de acuerdo al material de manufactura\footnote{El motor se asume como una masa uniformemente distribuida.}
El tensor de inercia de cada objeto fue medido respecto a su centro de masa.
Empleando el teorema de ejes paralelos (\textbf REFERENCIA) se trasladó el efecto de los tensores de inercia necesarios al marco referencial local.
A continuación se presentan los valores correspondientes a cada objeto de interés para la simulación.



\subsection{Plataforma móvil}

% Px = convertGramMM2toKgM2 * 7246289714.88;
% Py = convertGramMM2toKgM2 * 7246289714.88;
% Pz = convertGramMM2toKgM2 * 13887070726.86;
% 
% Ix = [1; 0; 0];
% Iy = [0; 1; 0];
% Iz = [0; 0; 1];

\begin{table}[hb!]
 \begin{center}
\begin{tabular}{lclc}
 $ I_x $ & $ [1 \ 0 \ 0]^T $ & $ P_x $ & 7246289714.88\\
 $ I_y $ & $ [0 \ 1 \ 0]^T $ & $ P_y $ & 7246289714.88\\
 $ I_z $ & $ [0 \ 0 \ 1]^T $ & $ P_z $ & 13887070726.86
\end{tabular}
\end{center}
\caption{Datos de inercia de la plataforma móvil.}
\end{table}


\subsection{Base}

% Px = convertGramMM2toKgM2 * 348997674.08;
% Py = convertGramMM2toKgM2 * 349004086.28;
% Pz = convertGramMM2toKgM2 * 696714510.70;
% 
% Ix = [0.73; -0.68; 0];
% Iy = [0.68; 0.73; 0];
% Iz = [0; 0; 1.0];

\begin{table}[hb!]
 \begin{center}
\begin{tabular}{lclc}
% \begin{multicolumn}{2}{c}{Vector} & \begin{multicolumn}{2}{c}{Inertia [$g/mm^2$]} \\
 $ I_x $ & $ [1 \ 0 \ 0]^T $ & $ P_x $ & 348997674.08\\
 $ I_y $ & $ [0 \ 1 \ 0]^T $ & $ P_y $ & 7246289714.88\\
 $ I_z $ & $ [0 \ 0 \ 1]^T $ & $ P_z $ & 13887070726.86
\end{tabular}
\end{center}
\caption{Datos de inercia de la plataforma móvil.}
\end{table}

\subsection{Actuador prismático con motor}
\begin{table}[hb!]
 \begin{center}
\begin{tabular}{lclc}
% \begin{multicolumn}{2}{c}{Vector} & \begin{multicolumn}{2}{c}{Inertia [$g/mm^2$]} \\
 $ I_x $ & $ [1 \ 0 \ 0]^T $ & $ P_x $ & 7246289714.88\\
 $ I_y $ & $ [0 \ 1 \ 0]^T $ & $ P_y $ & 7246289714.88\\
 $ I_z $ & $ [0 \ 0 \ 1]^T $ & $ P_z $ & 13887070726.86
\end{tabular}
\end{center}
\caption{Datos de inercia de la plataforma móvil.}
\end{table}

\subsection{Junta esférica con tornillo}
\begin{table}[hb!]
 \begin{center}
\begin{tabular}{lclc}
% \begin{multicolumn}{2}{c}{Vector} & \begin{multicolumn}{2}{c}{Inertia [$g/mm^2$]} \\
 $ I_x $ & $ [1 \ 0 \ 0]^T $ & $ P_x $ & 7246289714.88\\
 $ I_y $ & $ [0 \ 1 \ 0]^T $ & $ P_y $ & 7246289714.88\\
 $ I_z $ & $ [0 \ 0 \ 1]^T $ & $ P_z $ & 13887070726.86
\end{tabular}
\end{center}
\caption{Datos de inercia de la plataforma móvil.}
\end{table}


% Se incluye como anexo los planos de la antena parabólica empleados para el 
% estudio de la dinámica
% de plataforma Gough Stewart, cortesía de la compañía \emph{radiowaves}\footnote{www.radiowaves.com}.



% \includepdf[pagecommand={}, scale=1.0, pages={1,2,3}]{specs.pdf}


% \appendix
% \section{MATLAB}
\newpage
 \begin{lstinputlisting}{./code/Vectores.m}
   
%    This is a test.



\end{document}
