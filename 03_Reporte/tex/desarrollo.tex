La PGS puede ser modelada en un inicio por medio de su cinemática inversa. Observando el modelo en la figura (insertar figura de la plataforma) se puede formular lo siguiente:
\begin{equation} \label{plat_grl}
p_i = d + Ra_i = b_i + l_i
\end{equation}

Siendo $Ra_i$ el lugar donde un extremo del actuador en la plataforma es colocado, el valor $b_i$ es el lugar donde el otro extremo del actuador es colocado en la base y $d$ es la distancia que debe de moverse la plataforma respecto de la base. Definimos $R$ como la matriz de rotación extrínseca de la plataforma respecto a la base. 
\begin{equation}
R = R_zR_yR_x = R_{xyz}
\end{equation}
El actuador es un pistón controlado por el largo y tomando en cuenta que el valor del actuador es $l_i$, podemos reescribir la ecuación \ref{plat_grl} de la siguiente manera.
\begin{equation}
l_i = d + Ra_i - b_i
\end{equation}
El vector $l_i$ se obtiene como coordenadas $[x\ y\ z]^T$ al cual se debe aplicar la norma para obtener la dimensión del actuador y será la coordenada generalizada del pistón i-ésimo de la plataforma.
\begin{equation}\label{eq_coordgral}
q_i = ||l_i|| = \sqrt{l_i^Tl_i}
\end{equation}
De la ecuación (\ref{eq_coordgral}) se obtendrá el jacobiano para después despejar el twist y utilizar los valores de velocidad lineal y angular de la plataforma para el desarrollo del control por fuerzas. Para obtener el jacobiano se plantea la siguiente ecuación
\begin{equation} \label{equgral_q}
J\dot{q}=\nu
\end{equation}
La ecuación \ref{eq_coordgral} al ser derivada respecto al tiempo es parecido a la ecuación del jacobiano en la solución de las velocidades de las coordenadas generalizadas.

\begin{equation}
\frac{d}{dt}q = \frac{d}{dt}||l_i|| = \frac{d}{dt}\sqrt{l_i^Tl_i} 
\end{equation}
La ecuación anterior se desarrolla para tener la forma del jacobiano inverso:
\begin{equation}
\dot{q}=J^{-1} \nu = A \begin{bmatrix}
v_p\\
\omega
\end{bmatrix} \Rightarrow A = J^{-1}
\end{equation}

Desarrollamos la derivada de $||l_||i$:
\begin{equation}
\dot{q} = \frac{1}{2||l_i||} \dot{l_i} \cdot l_i + l_i \cdot \dot{l_i} = \frac{1}{||l_i||} \dot{l_i} \cdot l_i \\ \bigskip
\end{equation}

\begin{equation}
\dot{q}=\frac{1}{||l_i||}(\dot{d} + [\omega \times] Ra_i)\cdot(d + Ra_i -b_i) 
\end{equation}
\begin{equation*}
= \frac{1}{||l_i||}(v_p - [(Ra_i)\times]\omega)(l_i)
\end{equation*}

\begin{equation}
\dot{q} = v_p \cdot l_i - [(Ra_i)\times]\omega \cdot l_i 
\end{equation}
\begin{equation*}
= v_p \cdot l_i + [(Ra_i)\times]l_i \cdot \omega
\end{equation*}

\begin{equation}
\dot{q} = \frac{1}{||l_i||} v_p \cdot l_i + [(Ra_i)\times]l_i \cdot \omega
\end{equation}
\begin{equation} \label{jac_inv}
\dot{q} = \frac{1}{||l_i||} [l_i^T , [(Ra_i)\times]l_i^T] \begin{bmatrix}
v_p\\
\omega
\end{bmatrix}
\end{equation}
Con el desarrollo se encuentra que la jacobiana inversa parte de la ecuación \ref{jac_inv} y se define como:
\begin{equation}\label{jac_A}
A = J^{-1} = \begin{bmatrix}
\vec{u_i}^T & [(Ra_i)\times]\vec{u_i}^T
\end{bmatrix}
\end{equation}
Al invertir la matriz $A$ de la ecuación \ref{jac_A} se obtiene la jacobiana de la PGS, con la cual se obtendrán las velocidades lineales y angulares de la PGS.
\begin{equation*}
J = \begin{bmatrix}
\vec{u_i}^T & [(Ra_i)\times]\vec{u_i}^T
\end{bmatrix}^{-1}
\end{equation*}
Utilizando la Jacobiana y desarrollando la ecuación \ref{equgral_q} se obtienen las velocidades lineales y angulares de la PGS.
\begin{equation*}
\begin{bmatrix}
\vec{u_i}^T & [(Ra_i)\times]\vec{u_i}^T
\end{bmatrix}^{-1} \dot{q} = \begin{bmatrix}
v_p\\
\omega
\end{bmatrix}
\end{equation*}
Para el desarrollo de la dinámica del robot, se conoce que soportara en la plataforma un disco parabólico con masa de 250 lb. Para obtener el modelo dinámico de la PGS con el disco hay se necesitan los valores de velocidad lineal y angular con respecto de la plataforma, se plantea la siguiente transformación.
\begin{equation} \label{equ_tr0-1}
\begin{split}
M_{6x6}\
\begin{bmatrix}
v_p^0\\
\omega^0
\end{bmatrix}  = \begin{bmatrix}
v_p^1\\
\omega^1
\end{bmatrix}
\end{split}
\end{equation}
En la ecuación \ref{equ_tr0-1} se identifica que al multiplicar las velocidades respecto a la base de la PGS se obtienen los valores de velocidades respecto a la plataforma. La matriz $M$ debe de tener la siguiente forma:
\begin{equation}
M = \begin{bmatrix}
I & 0_{3x3} \\
0_{3x3} & R^T
\end{bmatrix}
\end{equation}
Donde el primer bloque es la matriz identidad debido a que las velocidades lineales tanto de la base como de la plataforma son iguales, la matriz $E$ es la transformación de las velocidades angulares y tiene la siguiente forma:
\begin{equation*}
R^T = \begin{bmatrix}
R_{11} & R_{21} & R_{31}\\
R_{12} & R_{22} & R_{32}\\
R_{13} & R_{23} & R_{33}
\end{bmatrix}
\end{equation*}
Con la obtención del twist respecto a la plataforma del robot se pueden desarrollar las ecuaciones de energía. Se obtiene la energía potencial por medio de la altura del disco parabólico $dp$ sobre la plataforma del robot.
\begin{equation}
\begin{split}
P_{dp} = m_{dp}gh = [0\ 0\ m_{dp}g] \begin{bmatrix}
P_x\\
P_y\\
P_z\\
\end{bmatrix}\\
\\
P_z = d + R\begin{bmatrix}
0\\
0\\
cm_{dp}\\
\end{bmatrix}\\
\end{split}
\end{equation}
El punto $P_z$ es la altura del disco parabólico respecto a la base de la PGS. La energía cinética depende de las velocidades sobre el disco parabólico y se define
\begin{equation}
K = \frac{1}{2} \nu m_{dp} \nu = \frac{1}{2} \dot{q}^T J^T\ m\ J \dot{q}
\end{equation}
