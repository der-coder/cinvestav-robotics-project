% \section{Desarrollo}
\section{Parámetros de diseño}

\subsection{Restricciones}
El robot paralelo está restringido en su movimiento 
al ser una cadena cerrada.
A su vez, el diseño del mismo impone restricciones dimensionales 
que deben ser tomadas en cuenta para su análisis.
Estas restricciones surgen de los parámetros de diseño que 
se seleccionaron para el modelo del robot y el simulador.
La tabla \ref{tab: restricciones} muestra los 
parámetros de diseño establecidos.

\begin{figure}[htb!]
 \centering
 \includegraphics[width=8.5cm]{PGScrop.jpg}
 \caption{Plataforma Gough-Cappel.}
 \label{fig: cad}
\end{figure}


\begin{table}[htb!]
\centering
\begin{tabular}{lc}

\multicolumn{2}{c}{Parámetros de diseño} \\ \hline
Radio de la base ($r_b$) & $0.44205 \ [m]$ \\ 
Radio de la plataforma ($r_a$) & $0.36248 \ [m]$ \\ 
Separación entre juntas en la base ($k_b$) & $0.7 \ [m]$ \\ 
Separación entre juntas en la plataforma ($k_a$) & $0.2986 \ [m]$ \\ 
Longitud mínima del actuador ($q_{min}$) & $1.28929 \ [m]$ \\ 
Mangitud máxima del actuador ($q_{max}$) &  $2.19 \ [m]$ \\ 
\end{tabular}
\caption{Restricciones dimensionales del robot paralelo.}
\label{tab: restricciones}
\end{table}

Las restricciones presentes en la tabla \ref{tab: restricciones}
son utilizados para definir la cinemática del robot paralelo.
Los valores de $r_a$ y $r_b$ definen la 
distancia radial en la que cada junta debe ser 
posicionada respecto al centro de la plataforma y la base.
Las juntas empleadas para el robot se muestran 
en la figura \ref{fig: joints}

\begin{figure}[htb!]
 \centering
    \begin{subfigure}[b]{0.4\textwidth} 
    \centering
    \includegraphics[width=0.6\textwidth]{CARDAN.png}
        \caption{Junta universal o cardán.}
        \label{fig: junta universal}
    \end{subfigure}
    ~ %add desired spacing between images, e. g. ~, \quad, \qquad, \hfill etc. 
      %(or a blank line to force the subfigure onto a new line)
    \begin{subfigure}[b]{0.4\textwidth}
    \centering
         \includegraphics[width=0.6\textwidth]{BALL_JOINT.png}
        \caption{Junta esférica.}
        \label{fig: junta esferica}
    \end{subfigure}
    ~ %add desired spacing between images, e. g. ~, \quad, \qquad, \hfill etc. 
    %(or a blank line to force the subfigure onto a new line)
    \caption{Juntas empleadas en la plataforma Gough-Cappel}\label{fig: joints}
\end{figure}

Las junta universales (figura \ref{fig: junta universal}) están
instaladas en la base del robot. 
La junta universal permite rotaciones en dos ejes.
La configuración de las juntas determina el orden de las rotaciones.
Para el robot diseñado se presenta una rotación en el eje y primero.
Una rotación en el eje x ocurre despúes de la primer rotación.
La orientación de las juntas establece que el eje y 
del marco referencial local se oriente en dirección al centro de la base.

Las juntas esféricas (figura \ref{fig: junta esferica}) conectan la 
plataforma móvil con los actuadores prismáticos.
Estas juntas permiten rotaciones en los tres ejes del marco referencial
local.
El orden de las rotaciones se asigna de la siguiente manera:
primero una rotación en el eje z, seguido de la rotación del 
marco referencial en el eje x, y finalmente el marco 
referencial es rotado en y.

\subsection{Ángulos de desfase}

La ubicación ideal de las juntas en la base y la plataforma móvil 
es en tres puntos ubicados sobre los radios $r_a$ y $r_b$, 
a 120 grados de distancia angular de cada uno, 
cada punto siendo la ubicación de dos juntas.
Las restricciones del modelo real del robot paralelo no permiten esta
ubicación, ya que esta posición no considera las posibles obstrucciones
que cada componente puede tener sobre los otros.\\

A manera de evitar estas obstrucciones y las posibles 
colisiones que conllevarían, 
las distancias $k_a$ y $k_b$ son parámetros de diseño.
Estas distancias generan ángulos de desfase 
$\alpha_a$ y $\alpha_b$ respecto al ángulo ideal de ubicación.
Estos ángulos de desfase son calculados de la siguiente manera:

\begin{equation} \label{eq: azi-a}
\alpha_a = \arctan\left(\frac{k_a}{2r_a}\right)
\end{equation}
\begin{equation} \label{eq: azi-b}
\alpha_b = \arctan\left(\frac{k_b}{2r_b}\right)
\end{equation}

Los ángulos de desfase permiten determinar las nuevas posiciones angulares 
$\Psi$ de cada junta, tal que la distancia angular de cada par de juntas 
respecto a su posición ideal sea $\pm \alpha$. 
Determinando los valores de $\psi_a$ y $\psi_b$, es posible
obtener los vectores de posición $  a_i$ y $  b_i$ 
para el punto de unión de cada junta
medido respecto a los marcos referenciales locales de 
la base ($\Sigma_b$) y la plataforma móvil ($\Sigma_a$).


\begin{equation} \label{eq: p_b}
  b_i = \begin{bmatrix}
r_b\cos(\psi_{bi})\\
r_b\sin(\psi_{bi})\\
0\\
\end{bmatrix}
\end{equation}

\begin{equation} \label{eq: p_a}
  a_i = \begin{bmatrix}
r_a\cos(\Psi_{ai})\\
r_a\sin(\Psi_{ai})\\
0\\
\end{bmatrix}
\end{equation}

\section{Cinemática}

La cinemática del robot paralelo es obtenida del conocimiento de una posición $  d$
y una orientación $  \theta_p$ deseada del efector final.
La posición de la plataforma móvil está dada por el vector $  p = [x \ y \ z]^T$
La orientación es el vector $  \theta_p = [\psi \ \theta \ \phi]^T$, 
cuyos elementos son los ángulos de la plataforma medidos respect al marco referencial inercial.

\begin{figure}[htb!]
    \centering
    \import{./img/}{isometric.pdf_tex}
%     \import{./img/}{goughStewart.pdf_tex}
    \caption{Diagrama de la plataforma Gough-Cappel.}
    \label{fig: gough stewart diagram}
\end{figure}

\subsection{Cinemática inversa de posición}

La cinemática inversa del robot paralelo parte de la suma de vectores 
mostrados en la figura \ref{fig: gough stewart diagram}.
Se observa que el vector de posición $  p_i$ puede ser representado 
por dos sumas equivalentes. 
\begin{itemize}
 \item La suma del vector de posición de la plataforma $  d$ 
medido respecto al referencial inercial y el producto de 
la matriz de rotación extrínseca $  R$ y el vector de posición 
de la junta $  a_i$. 
 \item La suma del vector de posición $  l_i$ medido respecto 
 al referencial local de la junta y el vector de posición $  b_i$.
\end{itemize}

\begin{subequations} \label{eq: plat_grl}
 \begin{align}
      p_i & =   d +   R  a_i \\
      p_i & =   b_i +   l_i
 \end{align}
\end{subequations}


Los elementos de la matriz de rotación 
extrínseca $  R$ están en función del vector $  \theta_p$.
La matriz se representa de la siguiente manera:

\begin{equation} \label{eq: Mrot-P}
  R = \begin{bmatrix}
C_\psi C_\theta & -S_\psi C_\phi + C_\psi S_\theta S_\phi & S_\psi S_\phi + C_\psi S_\theta C_\phi\\
S_\psi C_\theta & C_\psi C_\phi + S_\psi S_\theta S_\phi & -C_\psi S_\phi + S_\psi S_\theta C_\phi\\
-S_\theta & C_\theta S_\phi & C_\theta C_\phi\\
\end{bmatrix}
\end{equation}

El movimiento de la plataforma es determinado por el cambio de longitud
de los actuadores prismáticos $||  l_i||$.
Esta característica del sistema permite establecer
a las longitudes de los actuadores como las coordenadas 
generalizadas $q_i$ del sistema.
Cada coordenada generaliza se define entonces como la norma euclidiana
del vector $  l_i$ correspondiente.
A su vez, el vector $  l_i$ se obtiene al resolver 
\eqref{eq: plat_grl}.

\begin{equation} \label{eq: coord_grl}
    q_i = ||  l_i||
\end{equation}

\begin{equation} \label{eq: l}
    ||  l_i|| = \sqrt{ \langle  l_i,   l_i \rangle} 
\end{equation}

\begin{equation} \label{eq: largo_act}
  l_i =   d +   R   a_i -   b_i
\end{equation}

Utilizando las ecuaciones \ref{eq: coord_grl} y 
\ref{eq: largo_act} se obtiene el vector unitario 
$  \lambda_i$.

\begin{equation} \label{eq: vec_U}
  \lambda_i = \frac{  l_i}{q_i}
\end{equation}

\subsection{Pseudocinemática directa}

Para la obtención de la cinemática directas se propone la división de cada cadena paralela en 
cadenas seriales partiendo del marco referencial inercial.
A cada junta se asignan referenciales de movimiento rotacional  en sus respectivos ejes.
Esto permite el cálculo de cada cadena serial mediante 
transformaciones homogéneas (figura \ref{fig: cadena serial}).
Esta propuesta permite obtener un resultado fiable a pesar de la 
complejidad que presenta la obtención de la 
cinemática directa del sistema.

\begin{figure}[htb!]
 \centering
 \import{./img/}{frames.pdf_tex}
 \caption{Referenciales de la cadena serial.}
 \label{fig: cadena serial}
\end{figure}

Las transformaciones homogéneas $H_{ij}$ se presentan a
continuación:

\begin{subequations}
 \begin{align}
%   \sum_0^1 
  H_{i0} & = \begin{bmatrix}
R_z(\Psi_{bi}) & b_i\\
0 & 1
\end{bmatrix} \\
% \sum_1^2 
H_{i1} & = \begin{bmatrix}
R_y(\theta_{1i}) & 0\\
0 & 1
\end{bmatrix} \\
% \sum_2^3
H_{i2} & = \begin{bmatrix}
R_x(\theta_{2i}) & 0\\
0 & 1
\end{bmatrix} \\
% \sum_3^4 
H_{i3} & = \begin{bmatrix}
I_3 & q_{min}\\
0 & 1
\end{bmatrix}\\
% \sum_4^5
H_{i4} & = \begin{bmatrix}
R_z(\theta_{4i}) & l_i\\
0 & 1
\end{bmatrix} \\
% \sum_4^5 
H_{i5} & = \begin{bmatrix}
R_x(\theta_{5i}) & 0\\
0 & 1
\end{bmatrix} \\
% \sum_5^6 
H_{i6} & = \begin{bmatrix}
R_y(\theta_{6i}) & 0\\
0 & 1
\end{bmatrix} \\
% \sum_6^7
H_{i7} & = \begin{bmatrix}
R_z(\Psi_{ai}) & -Ra_i\\
0 & 1
\end{bmatrix}
 \end{align}
\end{subequations}

\begin{equation}
 \Psi_{bi}  = \psi_{bi} + \pi/2
\end{equation}

\begin{equation}
 \Psi_{ai}  = \psi_{ai} - \pi/2
\end{equation}

\subsection{Pseudocinemática inversa}

Habiendo asignado los referenciales a cada cadena serial
se definen variables 
$  \theta_i = [\theta_{1i} \ \theta_{2i} \dots \theta_{7i}]^T$ que describen el estado
de la cadena serial en función de los ángulos de orientación
y la coordenada generalizada $q_i$.
La metodología propuesta posee las siguientes restricciones:
\begin{itemize}
  \item Todas las cadenas seriales deben converger en el mismo punto.
  \item Las cadenas al converger en el punto también 
  deben converger en orientación.
  \item El único valor activo de la cadena es $q_i = \theta_{3i}$.
\end{itemize}

Los valores $\theta_{1i} \ \theta_{2i} \dots \theta_{7i}$
se obtienen de nuevas matrices de rotación $R_{i-1}^i$,
la coordenadas generalizada $q_i$ y el vector unitario
$  \lambda_i$.\\

La matriz de rotación $R(  \theta_i)$ se expresa como el producto de las 
matrices de rotación de la cadena serial.

\begin{equation}
R(  \theta_i) = R(  \theta_p)
\end{equation}

\begin{equation} \label{eq: th_12-46}
R(  \theta_i) = R(\Psi_{bi})R(\theta_{1i},\theta_{2i})R(\theta_{4i},\theta_{5i},\theta_{6i})R(\Psi_{ai})
\end{equation}

El vector unitario $  \lambda_i$ está orientado
de manera colinear con el eje z del referencial $\Sigma_1$. 
El vector $  \lambda_i$ medido en el referencial inercial 
% ($  \lambda_i^{(b_i)}$)
se expresa de la siguiente manera:

\begin{equation} \label{eq: th_12}
  \lambda_i^{(b_i)} = R^T_{z,\Psi_{bi}}   \lambda_i^{(0)} = R_{1}^{2} \  {\hat{k}}
\end{equation}

La matriz de rotación $R_1^2$ describe la rotación en dos ejes.
Primero une rotación en el eje y, seguido de una rotación en el eje x. 

\begin{equation}\label{eq: R_1_2}
R_1^2 = \begin{bmatrix}
C_{\theta_{1i}} & S_{\theta_{1i}} S_{\theta_{2i}} & S_{\theta_{1i}} C_{\theta_{2i}}\\
0 & C_{\theta_{2i}} & -S_{\theta_{2i}}\\
-S_{\theta_{1i}} & C_{\theta_{1i}} S_{\theta_{2i}} & C_{\theta_{1i}} C_{\theta_{2i}}
\end{bmatrix}
\end{equation}

Los componentes del vector $\lambda_i^{bi}$ se representan 
de la siguiente manera:
\begin{equation}\label{eq: lambda bi xyz}
  \lambda_i^{(bi)} = R^T_{z,\Psi_{bi}} \lambda_i^{(0)} = \begin{bmatrix}
C_{\Psi_{bi}} \lambda_{ix} + S_{\Psi_{bi}} \lambda_{iy} \\
-S_{\Psi_{bi}} \lambda_{ix} + C_{\Psi_{bi}} \lambda_{iy} \\
\lambda_{iz}
\end{bmatrix}
\end{equation}

Al evaluar \eqref{eq: R_1_2} en \eqref{eq: th_12} se obtiene el valor
de $  \lambda_i^{(bi)}$ en función de las coordenadas
$\theta_{1i}$ y $\theta_{2i}$.

\begin{equation}\label{eq: lambda bi theta}
\lambda_i^{(bi)} = \begin{bmatrix}
                        S_{\theta_{1i}} C_{\theta_{2i}}\\
                        -S_{\theta_{2i}}\\
                        C_{\theta_{1i}} C_{\theta_{2i}}
                    \end{bmatrix}
\end{equation}

Se resuelven las ecuaciones \eqref{eq: lambda bi xyz} y 
\eqref{eq: lambda bi theta} para determinar $\theta_{1i}$ y $\theta_{2i}$.

\begin{subequations}
 \begin{align}
    \theta_{1i} &= \arctan \left(\frac{C_{\Psi_{bi}} \lambda_{ix} + S_{\Psi_{bi}} \lambda_{iy}}{\lambda_{iz}}\right)\\
    \theta_{2i} &= \arcsin \left(S_{\Psi_{bi}} \lambda_{ix} - C_{\Psi_{bi}} \lambda_{iy}\right)
 \end{align}
\end{subequations}

Los valores de 
$\theta_{4i}$, $\theta_{5i}$ y $\theta_{6i}$ 
se obtienen de manera similar.

Se resuelve la ecuación \eqref{eq: th_12-46} para 
$R(\theta_{4i,5i,6i})$.

\begin{equation}\label{eq: resp_456}
    R(\theta_{4i,5i,6i})= R^T(\theta_{1i,2i})\ R^T(\Psi_{bi})\ R(\theta_i)\ R^T(\Psi_{ai})
\end{equation}

La matriz de rotación $R_4^6$ describe tres rotaciones: 
una rotación en z, una rotación en x, y una rotación en y.

\begin{subequations}
    \begin{align}
R_4^6 &= \begin{bmatrix}
         a & b & c\\
         d & e & f\\
         g & h & i
        \end{bmatrix}\label{eq: rot_456}\\
        % Rz Rx Ry
a &= C_{\theta_{4i}} C_{\theta_{6i}} - S_{\theta_{4i}} S_{\theta_{5i}} S_{\theta_{6i}}\\
b &= -C_{\theta{5i}} S_{\theta_{4i}}\\
c &= C_{\theta{4i}} S_{\theta_{6i}} + C_{\theta{6i}} S_{\theta_{4i}}\\
d &= C_{\theta{4i}} S_{\theta_{5i}} S_{\theta_{6i}} +C_{\theta{6i}} S_{\theta_{4i}} \\
e &= C_{\theta{4i}} C_{\theta{5i}}\\
f &= S_{\theta_{4i}} S_{\theta_{6i}} -C_{\theta{4i}} C_{\theta{6i}}\\
g &= -C_{\theta{5i}} S_{\theta_{6i}}\\
h &= S_{\theta_{5i}}\\ 
i &= C_{\theta{5i}} C_{\theta{6i}}
    \end{align}
\end{subequations}

Al igualar la matriz de rotación de la 
ecuación \eqref{eq: rot_456} con la matriz 
evaluada de la ecuación \eqref{eq: resp_456} 
se encuentran las siguientes soluciones:

\begin{equation}
\theta_{4i} = \arctantwo (-r_{12},r_{22})
\end{equation}
\begin{equation}
\theta_{5i} = \arcsin (r_{32})
\end{equation}
\begin{equation}
\theta_{6i} = \arctantwo (-r_{31},r_{33})
\end{equation}

Donde $r_{ij}$ es el elemento correpondiente
a la i-ésima fila y la j-ésima columna.

\subsection{Cinemática inversa de velocidad}
La cinemática inversa de velocidad se puede 
obtener desarrollando la derivada de 
la ecuación \eqref{eq: coord_grl}.

\begin{equation} \label{eq: d qi}
\begin{split}
\frac{d}{dt}q_i & = \frac{d}{dt}\sqrt{  l_i^T    l_i}  \\
\dot{q_i} & = \frac{1}{2q_i} ( {\dot l_i} \cdot   l_i +   l_i \cdot  {\dot l_i})\\
 &= \frac{1}{q_i} ( {\dot l_i} \cdot   l_i)
\end{split}
\end{equation}

La derivada del vector $  l_i$ se obtiene derivando
\eqref{eq: largo_act}.

\begin{equation}\label{eq: d li}
  {\dot l_i} =  {\dot d} + [  \omega \times] R  a_i
\end{equation}

Substituyendo \eqref{eq: d li} en \eqref{eq: d qi} y 
simplificando se obtiene una relación entre
la derivada respecto al tiempo de la coordenada generalizada 
y el twist  $\nu_p$ de la plataforma.

\begin{equation}\label{eq: d q def}
 \begin{split}
    \dot q  & =\frac{1}{q_i}( {\dot{d}} + [  \omega \times] R   a_i)\cdot   l_i   \\
    \dot q        & = \frac{1}{q_i}(  v_p - [(Ra_i)\times]\omega)\cdot l_i\\
    \dot q        & = \frac{l_i}{q_i}\left(   v_p - [(Ra_i)\times]\omega \right) \\
    \dot q        & =  \lambda_i \cdot   v_p - \lambda_i \cdot ([(Ra_i)\times] \omega)
 \end{split}
\end{equation}

Se introduce el concepto del vector extendido \emph{twist}. 
Se define de la siguiente manera \cite{olguin20183d}:

\begin{equation}\label{eq: def twist}
 \nu_p = \begin{bmatrix}
          v \\
          \omega
      \end{bmatrix}
\end{equation}

Con \eqref{eq: def twist} es posible 
reescribir \eqref{eq: d q def} como el producto de dos 
matrices.

\begin{equation} \label{eq: jac_inv}
\dot{q} = [\lambda_i^T, \\ -\lambda_i^T [(Ra_i)\times]\ ] \begin{bmatrix}
v_p\\
\omega
\end{bmatrix}
\end{equation}

Se observa que \ref{eq: jac_inv} puede 
representarse como:

\begin{equation} \label{eq: jac_g}
\dot{q} = A(d,R) \nu_p^{(0)}
\end{equation}

Donde la matriz $A$ es el inverso del 
\emph{Jacobiano geométrico} 
$J_g^{(0)}$ de la plataforma 
visto desde el referencial inercial. 

\begin{equation}
 A^{-1} = J_g^{(0)}
\end{equation}

El Jacobiano geométrico mapea las velocidades 
de las coordenadas generalizadas al twist de 
la plataforma.
A su vez, existe un Jacobiano analítico $J_a$ que mapea
estas mismas velocidades a la derivada de la pose
de la plataforma.

\begin{equation} \label{eq: jac_a}
\dot{z} = J_a(\cdot)\dot{q}
\end{equation}

De manera similar, la derivada de la pose $\dot z$
y el twist de la plataforma están relacionados 
a través de un operador cinemático $J_z$.

\begin{equation}\label{eq: kinematic operator mapping}
\nu_p^{(0)} = J_z\dot{z}
\end{equation}

\begin{equation}
 J_z = \begin{bmatrix}
        I_3 & 0\\
        0 & J_\theta
      \end{bmatrix}
\end{equation}

Este operador cinemático se expresa respetando 
el formato de alabeo, cabeceo y guiñada 
\cite{olguin20183d}.

\begin{equation}\label{eq: kinematic operator}
 J_\theta = \dfrac{\partial \omega^{(0)}}{\partial \dot \theta}= \begin{bmatrix}
        C_\theta C_\psi & -S_\psi & 0\\
        C_\theta S_\psi & C_\psi & 0\\
        -S_\theta & 0 & 1
      \end{bmatrix}
\end{equation}

Substituyendo \eqref{eq: kinematic operator mapping}
en \eqref{eq: jac_g} se obtiene la relación entre
la derivada de las coordenadas generalizadas y 
la derivada de la pose de la plataforma.

\begin{equation} \label{eq: q_twist}
\dot{q} = A(d,R)^0J_z(\theta) \dot{z}
\end{equation}

\subsection{Pseudocinemática directa de velocidad}

\subsubsection{Transformación de Plücker}

Se introduce el operador $\chi$, 
también conocido como transformación de Plücker.
Este operador hace un mapeo de vectores 
espaciales de movimiento de un marco referencial padre a un 
marco referencial hijo.
El operador se define de la siguiente manera \cite{olguin20183d}:

\begin{subequations}
 \begin{align}
  \chi & \triangleq \mathcal R^T(R_{j-1}^j) \mathcal T(d_{j/j-1}^{(j-1)})\label{eq: chi operator}\\
  \mathcal R(R) & \triangleq \begin{bmatrix}
                              R & 0\\
                              0 & R
                             \end{bmatrix}\\
  \mathcal T(  d) & \triangleq   \begin{bmatrix}
                                        I_3 & -[  d \times]\\
                                        0 & I_3
                                      \end{bmatrix}
 \end{align}
\end{subequations}

\begin{equation}\label{eq: plucker mapping}
 \nu_c^{(1)} = \chi(R, r_c) \nu^{(0)}
\end{equation}


Conociendo las condiciones iniciales de cada cadena serial
y empleando la transformación de Plücker
es posible hacer un mapeo recursivo del twist
de la plataforma.
El movimiento local 
$\nu_{i/i-1}^{i}$ depende solamente de 
$  \theta_i$.
Considerando la condición \emph{screw}
es posible definir 
$\nu_{i/i-1}^{i} = s_i \dot{\theta}_i$.
Se define el cálculo recursivo del \emph{twist} 
de la siguiente manera:

\begin{equation} \label{eq: tiwst_rec}
\nu_i = \chi_i(\theta_i)\nu_i + s_i\dot{\theta_i}
\end{equation}

Extendiendo los conceptos de traslación y rotación al twist 
$  \nu_i = J_i( {\theta_i})  {\dot \theta_i}$, 
es posible obtener el Jacobiano geométrico 
de la cadena serial de manera recursiva. 

\begin{equation}
J_i(\theta_i) = \chi(\theta_i) J_{i-1} + S_i
\end{equation}

Donde $S_i$ es una matriz de ceros 
($S_i \in R^{6\times n}$) excepto la columna 
\emph{i} que contiene al vector $s_i$ correspondiente 
al movimiento en el referencial actual $i$.\\
El Jacobiano inicial es una matriz nula $J_0 = 0$.

\subsubsection{Aceleración}

La aceleración de la cadena serial puede 
encontrarse derivando el twist $\nu_i$ en la ecuación 
\eqref{eq: tiwst_rec}.

\begin{equation}\label{eq: d twist}
\dfrac{d}{dt} \nu_i = \frac{d}{dt} (\chi_i(\theta_i)\nu_i + s_i\dot{\theta_i})
\end{equation}

La derivada de la transformación de Plücker 
se define como:
\begin{equation}\label{eq: d plucker}
\dot{\chi_i} = -\dot \theta_i \Omega(s_i) \chi_i
\end{equation}

El operador $\Omega$ se define de la siguiente forma
\cite{olguin2019multibody}, empleando la matriz 
$  j = [  j_1 \   j_2]^T$
como ejemplo.

\begin{equation}\label{eq: Omega}
 \Omega(  j) = \begin{bmatrix}
          [  j_2 \times] & [  j_1 \times] \\
          0 & [  j_2 \times]
          \end{bmatrix}
\end{equation}

Substituyendo \eqref{eq: d plucker} 
en \eqref{eq: d twist} 
y utilizando la propiedad del operador Omega
($\Omega(a) \ a = 0$)
\cite{olguin2019stewart}
se obtiene: 

\begin{equation}\label{eq: acceleration}
\dot \nu_i = \chi_i(\theta_i)\dot{\nu}_i - \dot \theta_i \Omega(s_i) \nu_i+ s_i\ddot{\theta_i}
\end{equation}

Dado el valor inicial de la aceleración 
$a_0 = G_0$ la ecuación \eqref{eq: acceleration} 
puede ser expresada como:
\begin{equation}
a_i = \chi_i(\theta_i) a_{i-1} - \dot{\theta_i}\Omega(s_i) \nu_i
\end{equation}

\section{Dinámica}

La dinámica del sistema se expresa mediante la ecuación 
de Euler-Lagrange generalizada empleando empleando la metodología 
de análisis de descomposición de cuerpos (BDA) \cite{olguin2019multibody}. 
Esta ecuación relaciona la matriz de inercia $H(q)$,
los efectos de Coriolis $C(q,\dot q)$, el vector de
gravedad $g(q)$ y las fuerzas disipativas $\tau_D$ con 
las fuerzas generalizadas $\tau$ aplicadas en el sistema.

\begin{subequations}\label{eq: lagrangiano_modelo}
\begin{align}
    H(  q) {\ddot{q}} + C(  q, {\dot q})\dot{q}  +g(q) - \tau_D  = \tau \\
    \sum_{i=1}^N J_i^T(q)F_i = \tau\\
    H(  q)  = \sum_{j=1}^N J_j^T(  q) M_j J_j(  q)\\
    C(  q,  {\dot q})  = \sum_{j=1}^N J_j^T(  q) \left ( M_j \dot J_j (  q) {\dot q} - \Omega^T(  \nu_j) M_j   \nu_j \right )\\
    g(  q)  = - \sum_{j=1}^N J_j^T(  q) M_j   G_j\\
    \tau_D  = \sum_{j=1}^N J_j^T(  q)   F_{f_{j}}
\end{align}
\end{subequations}


La formulación Lagrangiana de la dinámica directa del sistema requiere 
la obtención de los Jacobianos del sistema y sus derivadas.
Este cálculo agrega una complejidad considerable a la obtención
de una solución.

Mediante la formulación de Kirchoff \cite{olguin20183d}
es posible expresar la dinámica del sistema de una manera alternativa. 
Partiendo de la energía cinética del sistema es posible expresar
a las fuerzas y momentos del sistema en función de $v$ y $\omega$.
Esta nueva formulación es denominada \emph{formulación quasilagrangiana}.

\begin{subequations}\label{eq: kirchoff}
 \begin{align}
  \dfrac{d}{dt} \dfrac{\partial K}{\partial   v} +   \omega \times \dfrac{\partial K}{\partial   v} &=   f^{(1)}\\
  \dfrac{d}{dt} \dfrac{\partial K}{\partial   \omega} +   v \times \dfrac{\partial K}{\partial   v}  +   \omega \times \dfrac{\partial K}{\partial   \omega} &=   n^{(1)}
 \end{align}
\end{subequations}

El wrench $F_i$ puede ser expresado con  una nueva formulación.
Esta nueva formulación expresa el wrench en función del twist y su derivada.

\begin{equation}
F_i = M_i(\dot{\nu_i} - G_i) - \Omega^T(\nu_i)M_i\nu_i - F_{f_{i}}
\end{equation}

El tensor de masas $M_i$ es un vector extendido que contiene en sus elementos al tensor de inercia $I_{c_j}$, la masa $m$ y el vector de distancia al centro de masa desde el marco referencial.

\begin{equation}
 M_j = \begin{bmatrix}
        m_j I_3 & -m_j [  r_{cm_j} \times]\\
        m_j [  r_{cm_j} \times] & I_{c_j} -m_j [  r_{cm_j} \times]^2
      \end{bmatrix}
\end{equation}

Mediante los Lagrangianos locales se puede mapear la derivada de las coordenadas generalizadas al twist local.

\begin{equation} \label{eq: twist_loc}
\nu_i = J_i(q)\dot{q}
\end{equation}

Para la formulación quasilagrangiana se comienza con la substitución de la ecuación 
\eqref{eq: q_twist} en la ecuación \eqref{eq: twist_loc}.

\begin{equation}
\nu_i = J_i(q) A(z) R(R) \nu_p = T_i(z) \nu_p
\end{equation}

Se define nuevamente la matriz de inercia ahora en función del twist $H_\nu$ y los términos no lineales son agrupados en $h_\nu$.

\begin{subequations}
 \begin{align}
  H_v(z) &= \sum_{i=1}^N T_i^T (z) M_i T_i(z) > 0\\
  h_v (z, \nu_p) &= \sum_{i=1}^N T_j^T (z) \begin{Bmatrix}
                                          M_j (\dot T(z, \dot z)\nu_p -G_j) \\
                                          - \Omega^T (\nu_j) M_j \nu_j - F_{e_j}
                                          \end{Bmatrix}
 \end{align}
\end{subequations}
La formulación quasilagrangiana queda expresada de la 
siguiente manera:

\begin{equation}\label{eq: quasi lagrangian}
  H_v(z)\dot{\nu_p}+h_v(z,\nu_p) = R^T(R) A^T(z)\tau
\end{equation}

Resolviendo la ecuación \eqref{eq: quasi lagrangian}
para $h(z,\nu_p)$ y agregando el vector de 
aceleración gravitacional 
$G_j(q) = [{R_0^j}^T(q)g_0, \ 0 ]^T$ a la ecuación, 
se obtiene la siguiente expresión para $h(z,\nu_p)$.

\begin{subequations}
\begin{align}
 h_v(z,\nu_p) & =  \sum_{i=1}^N T_i^T(z)  \left( M_i \Theta  - \Omega^T(\nu_i)M_i\nu_i - F_{e_i} \right)\\
 \Theta & = \left(\dot T_i (z,\dot z) - G_i \right)
\end{align}
\end{subequations}

Este nuevo modelo dinámico depende de únicamente de valores de la pose y \emph{twist} de la plataforma.
Esta formulación vuelve independiente al modelo de la cinemática directa del modelo Lagrangiano debido a no depender de las coordenadas y velocidades generalizadas.

\subsection{Energía}

El trabajo de una partícula que se traslada de un punto 1 a un punto 2 se define como la integral de la fuerza aplicada $f$ a lo largo de su trayectoria \cite{olguin20183d}.

\begin{equation}\label{equ:trabajo 1}
W_{1-2} = \int_1^2 \ f \cdot ds
\end{equation}

Observando la definición de trabajo en una ruta, se puede reescribir la ecuación \eqref{equ:trabajo 1} para poder ser realizada en el tiempo.

\begin{equation}\label{equ:trabajo}
W_{1-2} = \int_{t1}^{t2} \ f \cdot \frac{ds}{dt} \ dt = \int_{t1}^{t2} \ (f \cdot v) \ dt
\end{equation}

El producto de la fuerza y el vector de velocidad $v$ en la ecuación \eqref{equ:trabajo} es la definición de potencia.

\begin{equation} \label{equ:potencia}
P \triangleq \frac{d}{dt}W = f \cdot v = \langle f,v \rangle
\end{equation}

Para el robot paralelo la potencia es obtenida de las velocidades generalizadas.
Substituyendo $\dot q$ en \eqref{equ:potencia}:

\begin{equation}\label{eq: power gough}
P = f \cdot \dot{q}
\end{equation}

El trabajo realizado se obtiene al integrar la ecuación \eqref{eq: power gough}.

\begin{equation}\label{equ:trabajo-plat}
W = \int f \cdot \dot{q}\ dt
\end{equation}
