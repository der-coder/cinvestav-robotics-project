Uno de los manipuladores paralelos mas populares es la \emph{Plataforma Gough-Stewart} (PGS) de 6 gdl (grados de libertad) propuesto por Eric Gough en 1954 y mejorado por D. Stewart en 1965 con la intención de realizar un simulador de vuelo como una de las aplicaciones finales de la plataforma. El sistema consiste en una plataforma movil, una plataforma fija y seis brazos extensibles conectando las dos plataformas. \\
\\
Una estructura paralelea es considerada una cadena cinemática cerrada, los brazos están conectados del efector final al origen por medio de una conexión paralela. El desarrollo cinemático del sistema es complicado y para su solución se necesita del entendimiento general de la PGS y su forma de operación. La cinemática de un robot se puede dividir entre cinemática directa e inversa. La cinemática directa se centra en encontrar la posicion y orientación del efector final al modificar las dimensiones de los brazos y la cinemática inversa se centra en utilizar la posicion final del efector final para encontrar el valor longitudinal de los brazos.El desarrollo de la cinemática inversa de la PGS es sencilla de obtener, sin embargo el desarrollo de la cinemática directa es complicada debido a la necesidad de solucionar ecuaciones no lineales para la solución. \\
\\
De la misma forma, es necesario conocer el modelo dinámico del sistema el cual define la menra en la que las energías y fuerzas se comportan en el sistema y contiene al igual que la cinemática directa complicaciones de desarrollo debido a la estructura de ciclo cerrado y las relaciones entre las partes del sistema. El modelo dinámico puede ser obtenido por medio de alguno de los siguientes métodos: \emph{Euler-Lagrange, principio de trabajo virtual, D'Alembert-Lagrange}.
Para fines del documento se utilizará el método de D'Alembert-Lagrange para obtener el modelo dinámico de la PGS. 
Cualquier robot para poder tener la fuerza y precisión para realizar una acción debe de poder rechazar perturbaciones externas asi como mantener la estabilidad en una referencia deseada, inclusive  si la referencia está en constante cambio. La teoría de control propone diferentes maneras de llegar a la estabilidad por medio de la aplicación de gradientes de energía suficientes para mantener la estabilidad en la referencia deseada.\\
\\
Por medio del presente escrito definirá en un inicio la cinemática de la PGS y seguido se realizará el modelo dinámico de la plataforma. En el aparatdo III se observarán los resultados del modelo dinámico con dos diferentes leyes de control (PD y PID) para comparar el gasto energético de cada uno con respecto a la tarea que debe realizar el robot.