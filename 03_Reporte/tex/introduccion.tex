La plataforma Gough Stewart (PGS), dotada de seis grados de 
libertad (gdl), es uno de los manipuladores paralelos más 
populares. Propuesto en 1954 por Eric Gough \cite{pretoria} 
como una herramienta para pruebas en llantas de 
vehículos\cite{gough}, fue redescubierto una década después 
por D. Stewart en su publicación describiendo un sistema de 
seis grados de libertad diseñado como un simulador de vuelo 
\cite{stewart}.

Dentro del trabajo de Stewart, el sistema se distinguió de 
otros simuladores de vuelo por la ausencia de un eje fijo 
con respecto al suelo \cite{stewart}. Esta característica le 
ha permitido al mecanismo simular, hasta cierto límite, 
condiciones de vuelo y aterrizaje que otros sistemas fueron 
incapaces de realizar.

La plataforma Gough Stewart no se vió limitada en cuanto a 
sus aplicaciones. Múltiples ideas han sido propuestas por la 
comunidad científica y la industria. Ejemplos de esto 
incluye su uso como mesa estabilizadora \cite{pretoria}, 
robot para cirugía, e incluso su implementación como sistema 
universal de pruebas para llantas \cite{gough2}.

Los elementos que en conjunto forman la plataforma Gough 
Stewart pueden agruparse en tres grupos: elementos de 
control, elementos de movimiento y elementos fijos. Los 
elementos de control consideran todo componente empleado en 
el robot que permita la medición de las condiciones del 
sistema y aquellos que están dedicados a procesar dicha 
información y enviar instrucciones al resto del sistema para 
asegurar un funcionamiento adecuado. Los elementos de 
movimiento incluyen todos los componentes responsables de 
efectuar el desplazamiento de la plataforma superior del 
sistema. En esta categoría se encuentran los actuadores 
lineales que al operar en conjunto permiten a la plataforma
hacer uso de un rango de movimiento y precisión 
que distinguen al mecanismo. Los elementos fijos consideran
al resto de los elementos del sistema que no son responsables
del movimiento o del control del mismo. El más importante de
estos es la plataforma inferior, que permanece fija con 
el suelo y provee el soporte necesario al resto del sistema
para operar.

A diferencia de otros robots, la plataforma Gough Stewart 
es una estructura de naturaleza paralela. 
Esto indica que el movimiento del elemento de interés se
ve afectado simultáneamente por cada elemento impulsor 
del sistema y que estas aportaciones no son estudiadas en
secuencia. Una estructura paralela es considerada una cadena 
cinemática cerrada. Cada actuador conecta al efector final
con el origen con conexiones paralelas. 
El estudio y análisis de este comportamiento resulta más 
complejo que en mecanismos de naturaleza secuencial.

El estudio cinemático de un robot cualquiera se divide
en el estudio de la cinemática directa o el estudio de 
la cinemática inversa del mismo.
El estudio de la cinemática directa de un robot
emplea el conocimiento de las dimensiones de sus
elementos de movimiento para así predecir 
la posición y orientación
que el efector final exhibirá. 
Por el contrario, la cinemática inversa hace uso de 
una posición y orientación conocida del efector
final para determinar las dimensiones de los elementos 
de movimiento.
Para el caso de la plataforma Gough Stewart, 
el estudio de la cinemática directa resulta
ser un reto pues las ecuaciones de movimiento
no representan un sistema lineal. 
Esta característica reduce considerablemente
las herramientas disponibles para el análisis
cinemático del sistema.
Aunado a esta limitación, el estudio de la 
cinemática inversa del sistema es mucho más preferible.

De la misma forma, es necesario conocer el modelo dinámico 
del sistema. 
Éste define la manera en que las fuerzas y energías
presentes en el sistema se comportan.
La obtención de un modelo dinámico de la 
plataforma Gough Stewart
cuenta también con sus propias complicaciones
debido a su naturaleza como estructura de ciclo cerrado.

El modelo dinámico puede ser obtenido 
por medio de alguno de los siguientes métodos: 
\emph{Euler-Lagrange, principio de trabajo virtual, 
D'Alembert-Lagrange}. 
Este trabajo empleará el método de D'Alembert-Lagrange
para obtener el modelo dinámico de la plataforma 
Gough Stewart.

El objetivo de implementar una platafora Gough Stewart es
aprovechar la precisión de sus movimientos para 
controlar de esta manera el movimiento de un objeto de 
interés. El control del  movimiento requiere también
que el sistema sea capaz de rechazar perturbaciones
y la habilidad de mantener la estabilidad de la 
plataforma superior aún cuando la posición de referencia
esté cambiando. La teoría de control 
propone diferentes maneras de llegar a la estabilidad por 
medio de la aplicación de gradientes de energía suficientes 
para mantener la estabilidad en la referencia deseada.

Este trabajo presentará el estudio de la cinemática
 y la dinámica de la plataforma Gough Stewart. 
 Se presentará también una comparación de diferentes 
 estrategias de control del sistema a manera 
 de determinar la estrategia que sea más 
 eficiente en términos de energía.
