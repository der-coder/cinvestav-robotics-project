El desarrollo del simulador de la plataforma Gough Stewart
ha presentado diferentes retos hasta el momento.
Cada uno de ellos ha servido para comprender a 
mayor detalle
el comportamiento del robot y sus limitaciones.
Cada área de conocimiento empleada exhibió 
sus retos únicos, como lo fue la obtención de
la derivada y el determinante para la 
cinemática inversa de la PSG. 
En el área de simulación, la implementación de
evaluaciones para factibilidad y la implementación
de animaciones del sistema para evaluar
qué tan robusto era el algoritmo desarrollado.

Los pasos que restan para el proyecto incluyen la conclusión
del estudio de dinámica del sistema para la 
aplicación seleccionada, la cual depende de los
valores obtenidos en el estudio de la 
cinemática directa del sistema.
