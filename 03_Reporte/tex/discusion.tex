\section{Discusión}
Se describen algunos de los sucesos del proyecto.
Se detallan las observaciones hechas sobre los eventos ocurridos y
el análisis de las situaciones.

\subsection{Entendimiento del problema}


\subsection{Simulador en lazo abierto}
La implementación de las ecuaciones del robot paralelo en 
MATLAB tuvo varios sucesos.
La primer implementación del código hacía uso de tensores de 
inercia $I$ que no correspondían a los tensores reales del sistema.
Esto se debió a que solamente se estaban probando las
rutinas desarrolladas y su correcta ejecución.
El comportamiento de la plataforma en esta etapa difería 
de manera exorbitante de lo esperado.
La prueba de caída libre del sistema llevaba al sistema a caer en 
una dirección diferente a una línea recta colinear con el eje z.

Se asumió que este comportamiento errático se debía al uso de 
valores incorrectos en el simulador.
Por ello, se implementaron nuevas rutinas para 
crear los componentes generales del sistema.
Estas nuevas rutinas hacen uso de los valores obtenidos de la
geometría del robot diseñado (vea apéndice \ref{sec: inertia})
para proveer al simulador de datos adecuados para
el cálculo de las coordenadas generalizadas.

Una vez incluidos estos datos en el simulador, 
las rutinas fueron capaces de 
presentar resultados confiables.
A partir de este momento fue posible el desarrollo
de las estrategias de control y 
linearización del sistema.

\subsection{Implementación de estrategias de control}




\subsection{Implementación de resortes de fin de carrera}
