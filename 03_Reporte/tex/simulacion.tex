\section{Simulador}

Se implementó el modelo dinámico del robot paralelo en MATLAB y SIMULINK.
Se crearon rutinas para realizar los diversos algoritmos necesarios para obtener la información del sistema.
\begin{itemize}
    \item Inicialización. Genera los datos de masa, inercia y geometría del sistema.
    \item Controladores. En función de las coordenadas generalizadas determinan el valor de $\tau$ para introducir en la función principal.
    \item Plataforma Gough-Cappel. En función del valor de $\tau$ determina el cambio en la pose y twist del sistema en cada intervalo de tiempo.
    \item Pseudocinemática inversa. Dada una posición y orientación determina las coordenadas locales de cada cadena serial $\theta_i$.
    \item Pseudocinemática inversa de velocidad. Dada la posición, orientación y twist calcula la transformación de Plücker, Jacobianos y $\dot \theta_i = J_{6i}^{-1}\nu_p$
    \item Dinámica inversa por quasilagrangiano. Dada una pose, el twist y las fuerzas generalizadas invoca los algortimos de pseudocinemática inversa y determina $h_\nu$ y $\dot \nu_p = H_\nu^{-1}(\mathcal{R}^T A^T \tau - h_\nu)+G_p$
\end{itemize}

En SIMULINK se implementaron las estrategias de control y el simulador operando en lazo cerrado. 
La construcción del simulador en SIMULINK dió lugar a la implementación de nuevas rutinas para realizar las transformaciones necesarias de las coordenadas generalizadas, fuerzas generalizadas, twist y pose para el funcionamiento correcto del simulador.

