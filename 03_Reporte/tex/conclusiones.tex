

\subsection{Neftali Jonatán González}

De forma individual puedo concluir que la PGS requiere de un 
cambio de paradigma al momento de tener el acercamiento al 
problema. Regularmente se tiene un acercamiento de tipo 
iterativo, dónde la solución de algún parámetro nos lleva a 
otro resultado hasta obtener la ecuación que describa el 
parámetro de nuestro interés, con la PGS es diferente. 
Requerimos de resolver el problema con cierta simultaneidad, 
por tal motivo el álgebra lineal es una herramienta clave, 
dado que se trata de una cadena cerrada.

La interpretación física de cada operación es distinta a la 
interpretación de las operaciones tan usadas en métodos 
geométricos para cadenas seriales; en ocasiones, no es tan 
evidente la interpretación de las operaciones vectoriales  
para el caso de la PGS, ha sido bastante útil el simulador 
para poder entender que es lo que hace cada operación y de 
qué forma nos  acerca a la solución que deseamos.

\subsection{Enrique Benavidez Téllez}

De manera personal puedo definir como resultado actual que 
el desarrollo de una plataforma Gough-Stewart es un proceso 
diferente al proceso de desarrollo de un robot serial. 
Debido a que se necesita desarrollar la cinemática inversa 
primero y en base a esta cinemática se debe de obtener la 
directa para poder utilizar las velocidades del efector 
final. Encontré que la PGS es un sistema que es fácil de 
resolver de manera geométrica en base a una posición y 
orientación requerida pero difícil con respecto al valor de 
las coordenadas generalizadas por la complejidad de 
operaciones necesarias para no caer en singularidades al 
moverse. El análisis dinámico de la PGS debe de llevar los 
valores de velocidad lineal y angular que se obtienen de la 
cinemática directa del sistema. 

\subsection{Isaac Ayala Lozano}
El proceso completo de análisis de la plataforma
Gough Stewart resultó ser una actividad bastante rigurosa.
La introducción de coordenadas generalizadas agregó 
nuevos retos al análisis. La obtención de las derivadas 
del sistema requirió de herramientas computacionales
debido a su complejidad. 

En general, la plataforma Gough Stewart y su estudio 
son un reto bastante interesante en cuanto a la 
aplicación de conocimientos de robótica.
Siendo un robot paralelo, la contribución al movimiento
final por cada elemento del sistema no permite
ser analizado con la misma facilidad que un robot serial.
