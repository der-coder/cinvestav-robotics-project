\section{Conclusiones}
Al haber finalizado el proyecto, se comprobó por medio de los diferentes controles aplicados al sistema que el consumo energético aumentó al reducir el error a su convergencia en cero. 
Se observó que al agregar la parte integral al controlador PD la energía consumida aumentó. Sin embargo la precisión del sistema también aumentó, este incremento ocasionó que el sistema genere un sobretiro tanto en posiciones como en energía.

En el aspecto de modelado de la dinámica del sistema, la implementación de la metodología BDA para solucionar los múltiples obstáculos demostró ser una opción efectiva.
La posibilidad de simplificar las cadenas cerradas a manera de representarlas como cadenas seriales abiertas permite emplear más herramientas para obtener una solución al problema.



\subsection{Neftali Jonatán González}
El consumo energético resultó ser algo semejante a lo que suponía. La acción integral agrega algo de fuerza extra para eliminar una perturbación constante.
Ya que realiza una sumatoria del error, (ya sea positivo o negativo) debe tener un período en el que el error es negativo para poder contrarrestar la suma de los errores positivos al comienzo de la trayectoria.
En caso de que el valor de referencia tenga coordenadas mayores al punto inicial, este comportamiento particular produce sobreimpulsos y por tanto un gasto extra de energía. 
Esto último debido al sobreimpulso y a la fuerza extra para poder contrarrestar la perturbación constante.

Por otra parte, el control con compensación de gravedad pretende alterar las fuerzas generalizadas a los valores requeridos para llegar al punto de referencia sin sobreimpulso.
La eliminación de sobreimpulso (o sobretiro) se debe a la acción derivativa, que se encarga de disminuir la velocidad conforme se acerca el sistema al punto de referencia.
El problema de la compensación de gravedad es que solo considera la parte lineal del sistema, por tanto solamente se aproxima a la referencia. Es incapaz de alcanzarla.\\

En cuanto al PD, hace converger las variables hasta que $k_p\left(error\right) = $perturbación constante.

Además, era notorio que el sistema tiene puntos en los que la linealización tiene un espacio efectivo mayor que otros, puntos que yo llamo, "de mayor estabilidad" en general.
Los puntos que se encuentren verticalmente alineados al centro de la base del robot son más estables.
También había puntos más bajos que presentaban mayor estabilidad, ésto puede ser porque la restricción del final de carrera crea un punto de equilibrio.


\subsection{Enrique Benavides Téllez}
Ya habiendo terminado el proyecto, se puede decir que se logró llegar al objetivo principal del proyecto: la comparación del desempeño y gasto energético con diferentes leyes de control en el robot. 
Con los resultados obtenidos se encontró que así como una ley de control permite tener un mejor desempeño del robot, también aumenta la energía necesaria para ese desempeño.
Bajo el concepto de "mejor desempeño igual a mayor energía consumida", la estrategia de control ya depende de la aplicación del robot. 
Si se necesita que el robot mantenga alto desempeño, precisión en posición y velocidad, va a ser necesario que el consumo energético sea mayor.
En caso de no necesitar tales niveles de desempeño el consumo se puede reducir manteniendo errores en el desempeño.

Cada nueva versión del proyecto fue debida a errores en el algoritmo para obtener los valores de la plataforma. 
Se observó que existen cada vez más elementos; los cuales generan cambios en la plataforma. 
Lo mas problemático es que todos estos elementos suelen tener impacto también en los términos no lineales; 
los cuales generan errores en el estado transitorio de la plataforma y errores cuando se quiere llegar a una referencia muy lejos del punto de linealización.

Algo realmente interesante del sistema, es que al agregar la matriz diagonal de momentos de inercia polar del motor multiplicado por su relación al cuadrado, el sistema deja de tener un comportamiento dominado por la plataforma. 
Ahora es dominado por los motores.
Aún con esto, los controles implementados mantuvieron su comportamiento. El PD con un error estacionario, el PD + G con el error estacionario menor al PD y el PID sin error estacionario pero con una estado transitorio que genera sobretiro.

\subsection{Isaac Ayala Lozano}
El objetivo del proyecto fue alcanzado. 
Fue posible observar y comparar el comportamiento del sistema 
en diferentes. 
Comenzando por el simulador implementado sin los datos del modelo realizado en Solidworks, se pudo observar que el simulador era capaz de generar resultados.
Esto en sí fue un paso impresionante ya que todos nosotros carecíamos de la certeza para asumir que se pudiera llegar a un simulador funcional.

Cada nueva iteración del modelo agregó un nuevo reto y un nuevo nivel de complejidad. 
La adición de los detalles del robot diseñado alteraron el comportamiento completo del sistema. 
El simulador ya era capaz de producir resultados confiables, como el caso de caída libre.
Inmediatamente después, la introducción de limitaciones de diseño como el torque máximo que cada motor es capaz de producir y la condición de fin de carrera
cambiaron nuevamente la naturaleza del simulador.

Las estrategias de control también fueron evolucionando. 
Partiendo de la linealización del sistema, cada control requirió ser sintonizado.
También surgió la necesidad de crear nuevas rutinas para proveer al controlador con la nueva información que requería para producir la respuesta deseada.

La respuesta del sistema a cada estrategia de control fue similar a lo que el equipo asumió inicialmente debido al conocimiento que cada integrante tiene sobre teoría de control.
El control PD presentó un error en estado estable sin importar las ganancias asignadas.
El control PID alteró el comportamiento del sistema al introducir un sobretiro en su recorrido antes de estabilizarse.
El control PD+G fue un nuevo tema para mí, así que solamente se pudo asumir sería una mejora sobre el control PD sin conocer cuál sería dicha mejora.

