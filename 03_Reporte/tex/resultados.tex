
\section{Resultados}

Las estrategias de control implementadas fueron probadas con dos casos de prueba: bajo una referencia estática en un punto arbitrario y bajo el seguimiento de una trayectoria circular.

El caso bajo una referencia estática se evaluó el tiempo de estabilización y el error en estado estable y en el caso de seguimiento de trayectorias se evaluó qué tan robusto es el sistema y el comportamiento del mismo fuera del punto de linealización.

La trayectoria generada a seguir por la plataforma esta definida por las siguientes ecuaciones:

\begin{equation}
    P = \begin{bmatrix}
    0.1 \cos (\omega t) \\
    0.1 \sen (\omega t) \\
    1.5
    \end{bmatrix}
\end{equation}

\begin{equation}
    \dot{P} = \begin{bmatrix}
    -0.1 \omega \sen (\omega t) \\
    0.1 \omega \cos (\omega t) \\
    0
    \end{bmatrix}
\end{equation}

En la simulación de la trayectoria no se contemplan movimientos en los ángulos de alabeo, cabeceo y guiñada.\\

Los controles que fueron implementados generaron diferentes resultados de la plataforma al llegar a la estabilidad. Los resultados fueron los siguientes:

\subsection{Control PD}

\begin{figure}[h]
    \centering
    \includegraphics[width=0.4\textwidth]{posPD.png}
    \caption{Posición del sistema bajo un patrón de movimiento - PD.}
    \label{fig:PD position}
\end{figure}

\begin{figure}[h]
    \centering
    \includegraphics[width=0.4\textwidth]{errorPD.png}
    \caption{Error del sistema bajo un patrón de movimiento- PD.}
    \label{fig:PD error}
\end{figure}


\begin{figure}[h]
    \centering
    \includegraphics[width=0.4\textwidth]{posPDe.png}
    \caption{Posición del sistema bajo una referencia estática- PD.}
    \label{fig:PD positione}
\end{figure}

\begin{figure}[h]
    \centering
    \includegraphics[width=0.4\textwidth]{errorPDe.png}
    \caption{Error del sistema bajo una referencia estática - PD.}
    \label{fig:PD errore}
\end{figure}

En las figuras \ref{fig:PD position} y \ref{fig:PD error} se observa el comportamiento de la plataforma bajo control PD para una trayectoria de movimiento. Observando la figura \ref{fig:PD error} se visualiza que existe un error que permanece a pesar que la plataforma sigue los puntos de la trayectoria generada.

En las figuras \ref{fig:PD positione} y \ref{fig:PD errore} se observa el comportamiento de la plataforma bajo control PD para una referencia estática . Observando la figura \ref{fig:PD errore}  se visualiza que existe un error constante en los ejes X, Y y Z. Este error es muy similar al error del caso anterior, puesto que el control PD no tiene manera de corregir el error en el estado estable.

\subsection{Control PD+G}

\begin{figure}[h]
    \centering
    \includegraphics[width=0.4\textwidth]{posPDpG.png}
    \caption{Posición del sistema bajo un patrón de movimiento - PD+G.}
    \label{fig:PDG position}
\end{figure}

\begin{figure}[h]
    \centering
    \includegraphics[width=0.4\textwidth]{errorPDpG.png}
    \caption{Error del sistema bajo un patrón de movimiento - PD+G.}
    \label{fig:PDG error}
\end{figure}

\begin{figure}[h]
    \centering
    \includegraphics[width=0.4\textwidth]{posPDpGe.png}
    \caption{Posición del sistema bajo una referencia estática - PD+G.}
    \label{fig:PDG positione}
\end{figure}

\begin{figure}[h]
    \centering
    \includegraphics[width=0.4\textwidth]{errorPDpGe.png}
    \caption{Error del sistema bajo una referencia estática - PD+G.}
    \label{fig:PDG errore}
\end{figure}



\subsection{Control PID}

\begin{figure}[h]
    \centering
    \includegraphics[width=0.4\textwidth]{posPID.png}
    \caption{Posición del sistema bajo un patrón de movimiento - PID.}
    \label{fig:PID position}
\end{figure}

\begin{figure}[h]
    \centering
    \includegraphics[width=0.4\textwidth]{errorPID.png}
    \caption{Error del sistema bajo un patrón de movimiento - PID.}
    \label{fig:PID error}
\end{figure}

\begin{figure}[h]
    \centering
    \includegraphics[width=0.4\textwidth]{posPIDe.png}
    \caption{Posición del sistema bajo una referencia estática - PID.}
    \label{fig:PID positione}
\end{figure}

\begin{figure}[h]
    \centering
    \includegraphics[width=0.4\textwidth]{errorPIDe.png}
    \caption{Error del sistema bajo una referencia estática - PID.}
    \label{fig:PID errore}
\end{figure}

\subsection{Trabajo}

\begin{figure}[h]
    \centering
    \includegraphics[width=0.4\textwidth]{energiamovsubplot.png}
    \caption{Energía de los diferentes controles bajo patrón de movimiento}
    \label{fig:my_label}
\end{figure}

\begin{figure}[h]
    \centering
    \includegraphics[width=0.4\textwidth]{energiamov.png}
    \caption{Comparación de energía en los diferente controles bajo un patrón de movimiento}
    \label{fig:energiamov}
\end{figure}

\begin{figure}[h]
    \centering
    \includegraphics[width=0.4\textwidth]{Energiaestatica.png}
    \caption{Comparación de energía en los diferente controles bajo referencia constante}
    \label{fig:energiamov}
\end{figure}