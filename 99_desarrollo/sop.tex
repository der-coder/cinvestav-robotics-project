\documentclass[letterpaper, 12pt]{article}

\usepackage[spanish]{babel}

\usepackage[margin = .5in]{geometry}
\usepackage{graphicx}
\usepackage{amsmath}
\usepackage[table,xcdraw]{xcolor}
\usepackage{float}
\usepackage{import}

\begin{document}
\section{Desarrollo}
\subsection{Restricciones}
La PGS al ser un robot de cadena cerrada obtiene limitaciones 
derivadas de la naturaleza del robot así como restricciones 
derivadas de los elementos que integran la PGS.\\

INSERTAR IMAGEN DEL CAD\\

La PGS diseñada para el proyecto cuenta con los siguientes 
parametros de diseño:

\begin{table}[h]\label{tab:restricciones}
\centering
\begin{tabular}{|ll|}
\hline
\multicolumn{2}{|l|}{\cellcolor[HTML]{EFEFEF}Parametros diseño} \\ \hline
Radio base & m ($r_b$)\\ \hline
Radio plataforma & m ($r_a$)\\ \hline
Separacion entre juntas (base) & m ($k_b$)\\ \hline
Separacion entre juntas (plataforma) & m ($k_a$)\\ \hline
Largo min actuador & m ($q_{min}$)\\ \hline
Largo max actuador & m ($q_{max}$)\\ \hline
\end{tabular}
\caption{Restricciones dimensionales de la PGS}
\end{table}

Con los valores señalados previamente se determinan las restricciones 
dimensionales de la PGS para el desarrollo de la cinemática de la PGS.\\

Los primeros dos valores de la tabla \ref{tab:restricciones} señalan 
la distancia radial en donde se deben de colocar las juntas de la 
plataforma (elementos pasivos). Estas juntas son las siguientes:\\

IMAGEN DE LA JUNTA CARDAN\\

Una junta universal o Cardan en la base de la PGS donde se permite la 
rotación en dos ejes, a fin del sistema propuesto las rotaciones en esta junta 
se realizan en el orden de rotación en eje $y$ seguido de rotacion en eje $x$.\\

IMAGEN DE LA JUNTA ESFERICA\\

Una junta esférica en la plataforma de la PGS donde se permite la rotación 
en tres ejes, a fin del sistema propuesto las rotaciones en esta junta se 
realizan en el orden de rotación en eje $z$, rotación en eje $x$ finalizando en 
rotación en eje $y$.\\

La PGS ideal debe de tener el punto de unión de las juntas en el mismo lugar, este 
punto se encuentra localizado en el punto generado por el radio y una separación angular 
de 120° a lo largo del circulo formado por el radio. Sin 
embargo debido a las características de las juntas, necesitan estar separadas para 
realizar el seguimiento de movimiento sin generar conflictos o colisiones entre los 
actuadores entre la juntas de la base y plataforma. Estos valores de separación se van 
a utilizar para obtener el desfase angular con respecto al ángulo ideal de unión. Este 
desfase se calcula de la siguiente manera. \\

\begin{equation} \label{equ:azi-a}
\alpha_a = tan^{-1}\left(\frac{k_a}{2r_a}\right)
\end{equation}
\begin{equation} \label{equ:azi-b}
\alpha_b = tan^{-1}\left(\frac{k_b}{2r_b}\right)
\end{equation}

Al sumar y restar el angulo de desfase $\alpha$ a los ángulos ideales se obtienen 
los vectores de desfase angular $\Psi_b$ y $\Psi_a$. Utilizando $r_b$, $r_a$, 
$\Psi_b$ y $\Psi_a$ se obtienen los puntos cartesianos de unión de la junta 
con la plataforma y la base de la siguiente manera:

\begin{equation} \label{equ:p_b}
b_i = \begin{bmatrix}
r_b\cos(\Psi_{bi})\\
r_b\sen(\Psi_{bi})\\
0\\
\end{bmatrix}
\end{equation}
\begin{equation} \label{equ:p_a}
a_i = \begin{bmatrix}
r_a\cos(\Psi_{ai})\\
r_a\sen(\Psi_{ai})\\
0\\
\end{bmatrix}
\end{equation}

\subsection{Cinemática de la PGS}
Para la cinemática de la PGS es importante conocer la posición y orientación del 
efector final, el cual es la plataforma en donde se conectan los actuadores. Se 
define la posición de la plataforma como un vector $d = [x,y,z]^T$ y la orientación 
la matriz como los ángulos medidos del referencial inercial en la forma de 
$\theta_p = [\psi,\theta,\phi]^T$.

\subsubsection{Cinemática inversa de posición}

\begin{figure}[ht]
    \centering
    \import{./img/}{goughStewart.pdf_tex}
    \caption{Diagrama de la plataforma Gough Stewart. REVISAR IMAGEN}
    \label{fig: gough stewart diagram}
\end{figure}

Conociendo la posición y orientación de la PGS, se puede encontrar de manera sencilla 
la cinemática inversa. Observando el modelo en la figura \ref{fig: gough stewart diagram} 
buscando el valor de $p_i$ se puede formular lo siguiente:

\begin{equation} \label{equ:plat_grl}
p_i = d + Ra_i = b_i + l_i
\end{equation}

Siendo $Ra_i$ el punto en donde se coloca la junta esferica de la plataforma, $b_i$ 
el punto donde se coloca la junta Cardan,  $d$ la distancia entre el referencial 
inercial y el local en la plataforma. Se define $R$ como la rotacion extrínseca formada 
por los valores de $\theta_p$ con la siguiente forma:
\begin{equation} \label{equ:Mrot-P}
R = \begin{bmatrix}
C_\psi C_\theta & -S_\psi C_\phi + C_\psi S_\theta S_\phi & S_\psi S_\phi + C_\psi S_\theta C_\phi\\
S_\psi C_\theta & C_\psi C_\phi + S_\psi S_\theta S_\phi & -C_\psi S_\phi + S_\psi S_\theta C_\phi\\
-S_\theta & C_\theta S_\phi & C_\theta C_\phi\\
\end{bmatrix}
\end{equation}

Los movimientos de la plataforma se realizan por medio de actuadores lineales controlados 
por el largo del mismo. El largo del piston se define como la coordenada generalizada del 
sistema $q_i$ el cual se obtiene por medio de la norma euclideana del vector de largo 
$l_i$ obtenido al despejar $l_i$ de la ecuacion \ref{equ:plat_grl}.
\begin{equation} \label{equ:coord_grl}
q_i = \sqrt{<l_i,l_i>}
\end{equation}
\begin{equation} \label{equ:largo_act}
l_i = d + Ra_i - b_i
\end{equation}

Utilizando las ecuaciones \ref{equ:coord_grl} y \ref{equ:largo_act} se encuentra el vector unitario como:

\begin{equation} \label{equ:vec_U}
\lambda_i = \frac{l_i}{q_i}
\end{equation}

\subsubsection{pseudo-Cinemática Directa}
Debido a la complejidad en la cinemática directa de manera tradicional, se propone 
dividir cada una de las cadenas paralelas partiendo del referencial inercial al referencial 
local de la plataforma tomando cada cadena como un robot serial. Se asignan a cada una de 
las juntas referenciales de movimiento rotacionales en su respectivo eje, lo cual permite 
desarrollar una cadena serial que puede ser definida por medio de transformaciones 
homogeneas.\\

INSERTAR IMAGEN DE LOS REFERENCIALES DE UNA PATA\\

\begin{equation}
\sum_0^1 = \begin{bmatrix}
R_z(\Psi_{bi} + \pi/2) & b_i\\
0 & 1
\end{bmatrix}
\end{equation}

\begin{equation}
\sum_1^2 = \begin{bmatrix}
R_y(\theta_{1i}) & 0\\
0 & 1
\end{bmatrix}
\end{equation}

\begin{equation}
\sum_2^3 = \begin{bmatrix}
R_x(\theta_{2i}) & 0\\
0 & 1
\end{bmatrix}
\end{equation}

\begin{equation}
\sum_3^4 = \begin{bmatrix}
I_3 & q_{min}\\
0 & 1
\end{bmatrix}
\end{equation}

\begin{equation}
\sum_4^5 = \begin{bmatrix}
R_z(\theta_{4i}) & l_i\\
0 & 1
\end{bmatrix}
\end{equation}

\begin{equation}
\sum_4^5 = \begin{bmatrix}
R_x(\theta_{5i}) & 0\\
0 & 1
\end{bmatrix}
\end{equation}

\begin{equation}
\sum_5^6 = \begin{bmatrix}
R_y(\theta_{6i}) & 0\\
0 & 1
\end{bmatrix}
\end{equation}

\begin{equation}
\sum_6^7 = \begin{bmatrix}
R_z(\Psi_{ai} - \pi/2) & -Ra_i\\
0 & 1
\end{bmatrix}
\end{equation}

\subsubsection{pseudo-Cinemática inversa}
Al asignar los referenciales a las cadenas paralelas, se definen variables que dependen de 
valores angulares así como del valor $q_i$. Para el funcionamiento de esta metodología se 
deben de mantener las siguientes restricciones.

\begin{itemize}
  \item Todas las cadenas paralelas deben converger en el mismo punto.
  \item Las cadenas al converger en el punto también deben converger en orientación.
  \item El único valor activo de la cadena es $q_i = \theta_{3i}$.
\end{itemize}

Para encontrar los valores articulares de las juntas, se conoce el valor de $q_i$ y $\lambda_i$ de las ecuaciones \ref{equ:coord_grl} y \ref{equ:vec_U} de la cinemática inversa de la plataforma. El vector unitario $\lambda_i$ se encuentra a lo largo del eje $z$ del referencial $\sum_1$ y del referencial $\sum_4$. La cadena serial en orientación se puede escribir como:

\begin{equation} \label{equ:th_12-46}
R(\theta_i) = R(\psi_{bi}+\pi/2)R(\theta_{1i},\theta_{2i})R(\theta_{4i},\theta_{5i},\theta_{6i})R(\psi_{ai}-\pi/2)
\end{equation}

De la misma aseveración se puede definir
\begin{equation} \label{equ:th_12}
\lambda_i^{(b_i)} = R^T_{z,\psi_{bi}+\pi/2} \lambda_i^{(0)} =R_1^2 \hat{k}
\end{equation}

Como la configuración de $R_1^2$ es la matriz de rotación realizada en el eje $y$ seguido del eje $x$ se tiene la matriz de rotación:
\begin{equation}
R_1^2 = \begin{bmatrix}
C_{\theta_{1i}} & S_{\theta_{1i}} S_{\theta_{2i}} & S_{\theta_{1i}} C_{\theta_{2i}}\\
0 & C_{\theta_{2i}} & -S_{\theta_{2i}}\\
-S_{\theta_{1i}} & C_{\theta_{1i}} S_{\theta_{2i}} & C_{\theta_{1i}} C_{\theta_{2i}}
\end{bmatrix}
\end{equation}

Y el valor de $\lambda_i^{bi}$ se define:
\begin{equation}
\lambda_i^{(bi)} = R^T_{z,\psi_bi+\pi/2} \lambda_i^{(0)} = \begin{bmatrix}
C_{\psi_{bi}+\pi/2} \lambda_{ix} + S_{\psi_{bi}+\pi/2} \lambda_{iy} \\
-S_{\psi_{bi}+\pi/2} \lambda_{ix} + C_{\psi_{bi}+\pi/2} \lambda_{iy} \\
\lambda_{iz}
\end{bmatrix}
\end{equation}

Al reemplazar en la ecuación \ref{equ:th_12} se obtiene:
\begin{equation}
\lambda_i^{(bi)} = \begin{bmatrix}
S_{\theta_{1i}} C_{\theta_{2i}}\\
-S_{\theta_{2i}}\\
C_{\theta_{1i}} C_{\theta_{2i}}
\end{bmatrix}
\end{equation}

Y de esta igualación se obtiene que:
\begin{equation}
\theta_{1i} = \arctan\left(\frac{C_{\psi_{bi}+\pi/2} \lambda_{ix} + S_{\psi_{bi}+\pi/2} \lambda_{iy}}{\lambda_{iz}}\right)
\end{equation}
\begin{equation}
\theta_{2i} = \arcsen\left(S_{\psi_{bi}+\pi/2} \lambda_{ix} - C_{\psi_{bi}+\pi/2} \lambda_{iy}\right)
\end{equation}

Los valores angulares de $\theta_{4i}$,$\theta_{5i}$ y $\theta_{6i}$ se obtienen del despeje de $R_4^6$ de la ecuación \ref{equ:th_12-46}:
\begin{equation}\label{equ:resp_456}
R(\theta_{4i,5i,6i})= R^T(\theta_{1i,2i})\ R^T(\psi_{bi}+\pi/2)\ R(\theta_i)\ R^T(\psi_{ai}-\pi/2)
\end{equation}
\begin{equation*}
R(\theta_i) = R(\theta_p)
\end{equation*}

Donde la matriz de rotación $R_4^6$ se define en los ejes $z$, $y$ y $x$ de la siguiente manera:
\begin{equation}\label{equ:rot_456}
Matriz Rotacion 4-6
\end{equation}

Al igualar la matriz de rotación de la ecuación \ref{equ:rot_456} con la matriz evaluada de la ecuación \ref{equ:resp_456} se encuentran las siguientes soluciones:

\begin{equation}
\theta_{4i} = \arctan2 (-r_{12},r_{22})
\end{equation}
\begin{equation}
\theta_{5i} = \arcsen (r_{32})
\end{equation}
\begin{equation}
\theta_{6i} = \arctan2 (-r_{31},r_{33})
\end{equation}

\subsubsection{Cinemática inversa de velocidad}
La cinemática inversa de velocidad se puede obtener desarrollando la derivada de la ecuación \ref{equ:coord_grl}.

\begin{equation}
\frac{d}{dt}q_i = \frac{d}{dt}\sqrt{l_i^Tl_i} 
\end{equation}

Desarrollamos la derivada de $||l_||i$:
\begin{equation}
\dot{q_i} = \frac{1}{2q_i} (\dot{l_i} \cdot l_i + l_i \cdot \dot{l_i}) = \frac{1}{q_i} (\dot{l_i} \cdot l_i)
\end{equation}

\begin{equation}
\dot{q}=\frac{1}{q_i}(\dot{d} + [\omega \times] Ra_i)\cdot l_i 
\end{equation}

\begin{equation*}
= \frac{1}{q_i}(v_p - [(Ra_i)\times]\omega)\cdot l_i
\end{equation*}

\begin{equation}
\dot{q} = \frac{l_i}{q_i}\left( v_p - [(Ra_i)\times]\omega \right) 
\end{equation}
\begin{equation*}
= v_p \cdot \lambda_i + [(Ra_i)\times]\lambda_i \cdot \omega
\end{equation*}

\begin{equation} \label{equ:jac_inv}
\dot{q} = [\lambda_i^T\ ,\ [(Ra_i)\times]\ \lambda_i^T] \begin{bmatrix}
v_p\\
\omega
\end{bmatrix}
\end{equation}

Con la expresion de la ecuación \ref{equ:jac_inv} se encuentra que la forma corresponde a la forma
\begin{equation} \label{equ:jac_g}
\dot{q} = A(d,R) \nu_p
\end{equation}

Donde la matriz $A$ es el jacobiano geométrico inverso $J_g^{(0)}$ de la plataforma desde el referencial inercial. Y mapea las velocidades de las coordenadas generalizadas al twist de la plataforma.\\

Al derivar la pose de la plataforma, se obtiene que el jacobiano geometrico es el que mapea los cambios de las coordenadas geeralizadas en los cambios de la pose y se define como:

\begin{equation} \label{equ:jac_a}
\dot{z} = J_a(\cdot)\dot{q}
\end{equation}

Existe una relacion entre la derivada de la pose y el twist, se encuentra que la relación entre estos dos es un operador cinemático ($J_z$) el cual se obtiene dependiendo de la configuración de orientación utilizada.\\
\begin{equation}
\nu_p^{(0)} = J_z\dot{z}
\end{equation}

Al realizar la igualación de la ecuacion \ref{equ:jac_g} y \ref{equ:jac_a} se encuentra:

\begin{equation}
\dot{q} = A(d,R)^0J_z(\theta) \dot{z}
\end{equation}

\subsubsection{pseudo-Cinemática de velocidad}

\end{document}