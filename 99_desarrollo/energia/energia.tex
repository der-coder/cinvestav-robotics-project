\documentclass[letterpaper, 12pt]{article}

\usepackage[spanish]{babel}

\usepackage[margin = .5in]{geometry}
\usepackage{graphicx}
\usepackage{amsmath}
\usepackage{amssymb}
\usepackage[table,xcdraw]{xcolor}
\usepackage{float}
\usepackage{import}

\begin{document}
\section{Desarrollo}
\subsection{Energía}
El trabajo realizado por una partícula al moverse de un punto 1 al punto 2 se puede expresar por medio de la siguiente definición, la integral de un movimiento en una ruta se define como el producto escalar de la fuerza en la partícula y la ruta realizada: [REFERENCIA GOLDSTEIN/LIBRO OLGUIN]

\begin{equation}\label{equ:trabajo}
W_{1-2} = \int_1^2 \ f \cdot ds
\end{equation}

Observando la definición de trabajo en una ruta, se puede reescribir la ecuación \ref{equ:trabajo} para poder ser realizada en el tiempo. Para lograrlo, se incluye un diferencial de tiempo modificando al valor de la ruta, generando:

\begin{equation}\label{equ:trabajo}
W_{1-2} = \int_{t1}^{t2} \ f \cdot \frac{ds}{dt} \ dt = \int_{t1}^{t2} \ (f \cdot v) \ dt
\end{equation}

Esta ecuación define la potencia de la partícula como la derivada en el tiempo del trabajo.
[REFERENCIA LIBRO OLGUIN]

\begin{equation} \label{equ:potencia}
P \triangleq \frac{d}{dt}W = f \cdot v = <f,v>
\end{equation}

En el caso de la PGS, se puede observar que la coordenada generalizada es la que realiza el movimiento y por lo tanto es donde se puede encontrar la potencia del sistema utilizando las velocidades generalizadas. Se redefine la ecuacion \ref{equ:potencia}:

\begin{equation}
P = f \cdot \dot{q}
\end{equation}

Y el tarbajo de la plataforma se obtiene como la integral de la ecuacion anterior.

\begin{equation}\label{equ:trabajo-plat}
W = \int f \cdot \dot{q}\ dt
\end{equation}

\section{Resultados}
\subsection{Comparación energética}


\end{document}