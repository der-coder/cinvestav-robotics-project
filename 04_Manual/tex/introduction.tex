\chapter{Introducción al simulador GSP}

\section{Sobre el software}
El simulador \emph{Gough-Stewart Platform} (GSP por sus siglas en inglés) es un 
conjunto de herramientas desarrolladas para el ambiente de computación numérica 
MATLAB  de la compañía MathWorks. El paquete incluye el código fuente necesario 
para la ejecución del simulador, una interfaz gráfica ejecutable desde Matlab, 
% una copia de la licencia del software,
ejemplos de uso del paquete y una copia de este manual.

\section{Motivación}
El simulador GSP fue creado como una herramienta para la evaluación de 
algoritmos de control diseñados para tal mecanismo. La disponibilidad de un 
simulador permite al usuario evaluar el comportamiento del sistema y validar el 
diseño de su implementación en una fase temprana de desarrollo de proyectos.

\section{Sobre la plataforma Gough-Stewart}

La plataforma Gough-Stewart es un robot de seis grados de libertad diseñado para 
controlar de manera precisa el movimiento de un objeto de interés, el cual está 
instalado en la base móvil del robot. Ejemplos de aplicación incluyen el 
movimiento de antenas parabólicas para rastreo de satélites y
% TODO Add citations here
simuladores de vuelo para uso comercial.
% TODO Insert image here!


